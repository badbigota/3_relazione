% !TeX encoding = utf8
% !TeX program = pdflatex
% !TeXpellcheck = it_IT

\documentclass[a4paper,11pt,oneside]{article} 

\usepackage{relazioni}
\usepackage{imakeidx}
\usepackage{colortbl}
\usepackage{booktabs}
\usepackage{blindtext}
\usepackage{titletoc}
\usepackage{hyperref}
\usepackage{graphicx}
\usepackage{subcaption}
\usepackage{wrapfig}
\usepackage{geometry}
\usepackage{array}
\usepackage[export]{adjustbox}
\usepackage{multirow}
\usepackage{multicol}

\usepackage{colortbl}


\graphicspath{{Figure/}}
\DeclareSIUnit\micronnewton{\micro\metre\per\newton}




\begin{document}
\input{Front-matter/Frontespizio}

\tableofcontents
\addtocontents{toc}{~\hfill{Pagina}\par}
\contentsmargin{6em}
\dottedcontents{section}[1em]{\bigskip}{2em}{1pc}
\dottedcontents{subsection}[3em]{\smallskip}{3em}{1pc}
\dottedcontents{subsubsection}[5em]{\smallskip}{4em}{1pc}


\newpage

\section{Obiettivo}
L'obiettivo dell'esperienza è la verifica del comportamento elastico di un filo metallico, la stima della sua costante elastica $K$ e del relativo modulo di Young $E$.

\section{Apparato sperimentale}\label{section:apparato}

\begin{figure}[h!]
    \centering
    \caption{Diagramma composizione apparato sperimentale}
    \includegraphics[width=12.5cm]{ApparatoSperimentale.jpg}
    \label{fig:apparato_sperimentale}
\end{figure}
L'apparato sperimentale risulta così composto:
\begin{enumerate}
    \item Un cilindro metallico (acciaio, tungsteno o ottone) posto internamente ad un tubo protettivo a tutela di perturbazioni esterne. Lo stesso ha un'estremità fissata in maniera solidale alla struttura di supporto e nella parte finale presenta un disco forato, al fine di valutarne l'allungamento.
    \item Un minimetro a quadrante regolabile con sensibilità $S=\num{1e5} \si{m^{-1}}$ con sonda appoggiata al riferimento. E' presente inoltre un contatore interno al quadrante del minimetro che tiene conto dei giri completi compiuti dalla lancetta.
    \item Un dinamometro regolabile tramite una ghiera che, ruotando, varia la forza applicata al filo. L'unità di misura utilizzata è grammi peso. Il dinamometro ha come più piccola tacca di lettura  $\SI{100}{gp}$
    \item Una leva collegata al filo metallico e al dinamometro utilizzata per quadruplicare la forza esercitata.
\end{enumerate}
Nel complesso sono stati impiegati 12 estensimetri diversi, dei quali vengono riassunte le caratteristiche nella Tabella \ref{tab:caratteristiche_estensimetri}.

\begin{table}[h!]
	\centering
	\begin{tabular}{|l|c|c|c|}  
\hline
		    \multirow{2}{*}{Materiale} & \multirow{2}{*}{\#}& \multicolumn{1}{l|}{Lunghezza}        & \multicolumn{1}{l|}{Diametro}\\ 
		    &&$[\si{mm}] \pm\SI{2}{mm}$&$[\si{mm}] \pm\SI{1}{\percent}$\\  
\hline
\multirow{9}{*}{Acciaio} & {\cellcolor[rgb]{0.85,0.85,0.85}}1  & {\cellcolor[rgb]{0.85,0.85,0.85}}950  & {\cellcolor[rgb]{0.85,0.85,0.85}}0.33   \\
& 2 & 950 & 0.356  \\
& {\cellcolor[rgb]{0.85,0.85,0.85}}3  & {\cellcolor[rgb]{0.85,0.85,0.85}}950  & {\cellcolor[rgb]{0.85,0.85,0.85}}0.381  \\
& 4 & 950 & 0.305  \\ & {\cellcolor[rgb]{0.85,0.85,0.85}}5  & {\cellcolor[rgb]{0.85,0.85,0.85}}950  & {\cellcolor[rgb]{0.85,0.85,0.85}}0.3    \\
& 6 & 950 & 0.4                                        \\
& {\cellcolor[rgb]{0.85,0.85,0.85}}7  & {\cellcolor[rgb]{0.85,0.85,0.85}}950  & {\cellcolor[rgb]{0.85,0.85,0.85}}0.279  \\
& 8 & 600 & 0.279                                      \\
& {\cellcolor[rgb]{0.85,0.85,0.85}}9  & {\cellcolor[rgb]{0.85,0.85,0.85}}800  & {\cellcolor[rgb]{0.85,0.85,0.85}}0.279  \\
& 10 & 700 & 0.279                                      \\
\hline
Tungsteno & {\cellcolor[rgb]{0.85,0.85,0.85}}11 &{\cellcolor[rgb]{0.85,0.85,0.85}} 1000 &{\cellcolor[rgb]{0.85,0.85,0.85}} 0.25\\
\hline
Ottone                   & 12 & 1000 & 0.5\\ \hline
\end{tabular}
	\caption{Estensimetri impiegati}
	\label{tab:caratteristiche_estensimetri}
\end{table}



\section{Metodi acquisizione dati}
Si specifica che i dati relativi alla prima parte sono stati trascritti da registrazioni video riferiti al decimo estensimetro. I dati impiegati nell'analisi sistematica dei vari estensimetri sono invece stati direttamente forniti in tabelle.\\

Vengono qui esposti i due diversi metodi impiegati per raccogliere i dati nella prima parte.

\subsection{Primo metodo - Allungamento e Accorciamento}\label{sec:miglior_metodo_misura}%aumento di 100 in 100
Come operazione preliminare si è agito sulla ghiera dell'estensimetro, manovrandola in modo tale che segnasse una tensione $F_{letta}=\SI{200}{gp}$ così da mettere in tensione il filo di metallo analizzato. Si è poi sollecitata delicatamente la manopola della sonda del minimetro così da eliminare l'eventuale contributo di giochi meccanici ed infine si è impostata la ghiera del minimetro in modo che la posizione della lancetta maggiore coincidesse con lo zero del quadrante. Dopo aver annotato il valore letto sul minimetro, tendendo in considerazione che una rotazione completa della lancetta corrisponde una variazione di lunghezza pari a $\num{1} \si{mm}$, si è manovrata la ghiera del dinamometro portandola alla tacca successiva, ovvero aumentando la forza letta di $\SI{100}{gp}$. Avendo cura di non oltrepassare la tacca e di smuovere la manopola della sonda di volta in volta, si sono annotati tutti i valori segnati dal minimetro. La ghiera è stata manovrata fino ad arrivare ad una forza letta di $\SI{1200}{gp}$. A tal punto si è diminuita la tensione esercitata dal dinamometro, variando sempre la $F_{letta}$ di $\SI{100}{gp}$, e annotando i valori.\\

\subsection{Secondo metodo - Misure ripetute}%video 400 -1000
Riportato l'estensimetro con il dinamometro a $\SI{200}{gp}$ si sono eseguite misure ripetute. La procedura è consistita nel portare il dinamometro ad esercitare una tensione letta di $\SI{400}{gp}$, dapprima aumentandola a partire da $\SI{200}{gp}$, poi diminuendola a partire da $\SI{600}{gp}$. È stata prestata la massima attenzione nel non oltrepassare la tacca di $\SI{400}{gp}$.
Il procedimento è stato effettuato per 9 volte "in accorciamento" e 9 "in allungamento" in corrispondenza della tacca $\SI{400}{gp}$ per poi ripetere la presa dati in corrispondenza della tacca di $\SI{1000}{gp}$, variando "in accorciamento" fino a $\SI{800}{gp}$ e "in allungamento" fino a $\SI{1200}{gp}$.\\
E' opportuno precisare che l'analisi dei video è stata compiuta singolarmente: ogni operatore ha trascritto la lettura del minimetro corrispondente ad ogni misurazione, in modo da avere un campione più significativo e compiere una analisi statistica più accurata.


\section{Analisi dati}
%INTRO GENERICA
Nella prima parte l'analisi si è svolta esclusivamente sull'estensimetro numero 10 riportato in Tabella \ref{tab:caratteristiche_estensimetri}. Nella seconda parte invece si è svolta un'analisi sistematica su tutti gli estensimetri in acciaio. Viene inoltre riportata l'analisi degli estensimetri in tungsteno e in ottone.

\subsection{Verifica dell'elasticità meccanica e definizione miglior protocollo di misura}\label{sec:prima_parte}
Sono stati impiegati diversi metodi per il calcolo del coefficiente elastico del filo del decimo estensimetro, a seconda del metodo di raccolta dati.\\

\subsubsection*{Primo Metodo - Misure in allungamento e accorciamento}
\begin{table}[h!]
    \centering
    \begin{tabular}{|cc|c|c|c||c|c|c|}
        \hline
        & $F_{letta}$ & $\Delta x_{Op 1}$ & $\Delta x_{Op 2}$ &$ \Delta x_{Op 3}$ &$\overline{ \Delta x}$ & $\sigma_{\Delta x}$ & $\sigma_{\overline{\Delta x}}$\\ 
        & $[\si{gp}]$& $[\SI{e-6}{\meter}] $&$[\SI{e-6}{\meter}] $&$[\SI{e-6}{\meter}] $&$[\SI{e-6}{\meter}] $&$[\SI{e-6}{\meter}] $&$[\SI{e-6}{\meter}] $\\
        \hline
        \multicolumn{1}{|c|}{\multirow{11}{*}{\rotatebox[origin=c]{90}{Allungamento}}} & {\cellcolor[rgb]{0.85,0.85,0.85}}200&	{\cellcolor[rgb]{0.85,0.85,0.85}}-2&	{\cellcolor[rgb]{0.85,0.85,0.85}}-3&	{\cellcolor[rgb]{0.85,0.85,0.85}}-2&	{\cellcolor[rgb]{0.85,0.85,0.85}}-2.3&	{\cellcolor[rgb]{0.85,0.85,0.85}}0.6&	{\cellcolor[rgb]{0.85,0.85,0.85}}0.3\\
        \multicolumn{1}{|c|}{}&300&	223&	222&	225&	223.3&	1.5&	0.9\\
        \multicolumn{1}{|c|}{}&{\cellcolor[rgb]{0.85,0.85,0.85}}400&	{\cellcolor[rgb]{0.85,0.85,0.85}}431&	{\cellcolor[rgb]{0.85,0.85,0.85}}431&	{\cellcolor[rgb]{0.85,0.85,0.85}}430&	{\cellcolor[rgb]{0.85,0.85,0.85}}430.7&	{\cellcolor[rgb]{0.85,0.85,0.85}}0.6&	{\cellcolor[rgb]{0.85,0.85,0.85}}0.3\\
        \multicolumn{1}{|c|}{}&500&	660&	660&	660&	660.0&	0.0&	0.0\\
        \multicolumn{1}{|c|}{}&{\cellcolor[rgb]{0.85,0.85,0.85}}600&	{\cellcolor[rgb]{0.85,0.85,0.85}}876&	{\cellcolor[rgb]{0.85,0.85,0.85}}878&	{\cellcolor[rgb]{0.85,0.85,0.85}}877&	{\cellcolor[rgb]{0.85,0.85,0.85}}877.0&	{\cellcolor[rgb]{0.85,0.85,0.85}}1.0&	{\cellcolor[rgb]{0.85,0.85,0.85}}0.6\\
        \multicolumn{1}{|c|}{}&700&	1093&	1091&	1092&	1092.0&	1.0&	0.6\\
        \multicolumn{1}{|c|}{}&{\cellcolor[rgb]{0.85,0.85,0.85}}800&	{\cellcolor[rgb]{0.85,0.85,0.85}}1310&	{\cellcolor[rgb]{0.85,0.85,0.85}}1311&	{\cellcolor[rgb]{0.85,0.85,0.85}}1310&	{\cellcolor[rgb]{0.85,0.85,0.85}}1310.3&	{\cellcolor[rgb]{0.85,0.85,0.85}}0.6&	{\cellcolor[rgb]{0.85,0.85,0.85}}0.3\\
        \multicolumn{1}{|c|}{}&900&	1513&	1512&	1512&	1512.3&	0.6&	0.3\\
        \multicolumn{1}{|c|}{}&{\cellcolor[rgb]{0.85,0.85,0.85}}1000&	{\cellcolor[rgb]{0.85,0.85,0.85}}1749&	{\cellcolor[rgb]{0.85,0.85,0.85}}1749&	{\cellcolor[rgb]{0.85,0.85,0.85}}1748&	{\cellcolor[rgb]{0.85,0.85,0.85}}1748.7&	{\cellcolor[rgb]{0.85,0.85,0.85}}0.6&	{\cellcolor[rgb]{0.85,0.85,0.85}}0.3\\
        \multicolumn{1}{|c|}{}&1100&	1948&	1949&	1948&	1948.3&	0.6&	0.3\\
        \multicolumn{1}{|c|}{}&{\cellcolor[rgb]{0.85,0.85,0.85}}1200&	{\cellcolor[rgb]{0.85,0.85,0.85}}2170&	{\cellcolor[rgb]{0.85,0.85,0.85}}2170&	{\cellcolor[rgb]{0.85,0.85,0.85}}2170&	{\cellcolor[rgb]{0.85,0.85,0.85}}2170.0&	{\cellcolor[rgb]{0.85,0.85,0.85}}0.0&	{\cellcolor[rgb]{0.85,0.85,0.85}}0.0\\ 
        \hline\hline
        \multicolumn{1}{|c|}{\multirow{11}{*}{\rotatebox[origin=c]{90}{Allungamento}}} & {\cellcolor[rgb]{0.85,0.85,0.85}}1200&	{\cellcolor[rgb]{0.85,0.85,0.85}}2170&	{\cellcolor[rgb]{0.85,0.85,0.85}}2170&	{\cellcolor[rgb]{0.85,0.85,0.85}}2170&	{\cellcolor[rgb]{0.85,0.85,0.85}}2170.0&	{\cellcolor[rgb]{0.85,0.85,0.85}}0.0&	{\cellcolor[rgb]{0.85,0.85,0.85}}0.0\\
        \multicolumn{1}{|c|}{}&1100&	1949&	1950&	1950&	1949.7&	0.6&	0.3\\
        \multicolumn{1}{|c|}{}&{\cellcolor[rgb]{0.85,0.85,0.85}}1000&	{\cellcolor[rgb]{0.85,0.85,0.85}}1727&	{\cellcolor[rgb]{0.85,0.85,0.85}}1727&	{\cellcolor[rgb]{0.85,0.85,0.85}}1727&	{\cellcolor[rgb]{0.85,0.85,0.85}}1727,0&	{\cellcolor[rgb]{0.85,0.85,0.85}}0.0&	{\cellcolor[rgb]{0.85,0.85,0.85}}0.0\\
        \multicolumn{1}{|c|}{}&900&	1514&	1513&	1514&	1513.7&	0.6&	0.3\\
        \multicolumn{1}{|c|}{}&{\cellcolor[rgb]{0.85,0.85,0.85}}800&	{\cellcolor[rgb]{0.85,0.85,0.85}}1303&	{\cellcolor[rgb]{0.85,0.85,0.85}}1303&	{\cellcolor[rgb]{0.85,0.85,0.85}}1305&	{\cellcolor[rgb]{0.85,0.85,0.85}}1303.7&	{\cellcolor[rgb]{0.85,0.85,0.85}}1.2&	{\cellcolor[rgb]{0.85,0.85,0.85}}0.7\\
        \multicolumn{1}{|c|}{}&700&	1082&	1081&	1082&	1081.7&	0.6&	0.3\\
        \multicolumn{1}{|c|}{}&{\cellcolor[rgb]{0.85,0.85,0.85}}600&	{\cellcolor[rgb]{0.85,0.85,0.85}}869&	{\cellcolor[rgb]{0.85,0.85,0.85}}869&	{\cellcolor[rgb]{0.85,0.85,0.85}}869&	{\cellcolor[rgb]{0.85,0.85,0.85}}869,0&	{\cellcolor[rgb]{0.85,0.85,0.85}}0.0&	{\cellcolor[rgb]{0.85,0.85,0.85}}0.0\\
        \multicolumn{1}{|c|}{}&500&	655&	655&	655&	655,0&	0.0&	0.0\\
        \multicolumn{1}{|c|}{}&{\cellcolor[rgb]{0.85,0.85,0.85}}400&	{\cellcolor[rgb]{0.85,0.85,0.85}}430&	{\cellcolor[rgb]{0.85,0.85,0.85}}431&	{\cellcolor[rgb]{0.85,0.85,0.85}}430&	{\cellcolor[rgb]{0.85,0.85,0.85}}430.3&	{\cellcolor[rgb]{0.85,0.85,0.85}}0.6&	{\cellcolor[rgb]{0.85,0.85,0.85}}0.3\\
        \multicolumn{1}{|c|}{}&300&	219&	220&	220&	219.7&	0.6&	0.3\\
        \multicolumn{1}{|c|}{}&{\cellcolor[rgb]{0.85,0.85,0.85}}200&	{\cellcolor[rgb]{0.85,0.85,0.85}}2&	{\cellcolor[rgb]{0.85,0.85,0.85}}1&	{\cellcolor[rgb]{0.85,0.85,0.85}}2&	{\cellcolor[rgb]{0.85,0.85,0.85}}1.7&	{\cellcolor[rgb]{0.85,0.85,0.85}}0.6&	{\cellcolor[rgb]{0.85,0.85,0.85}}0.3\\
        \hline
    \end{tabular}
    \caption{Dati Grezzi con $\overline{\Delta x}$, $\sigma_{\Delta x}$, $\sigma_{\overline{\Delta x}}$}
    \label{tab:dati_grezzi_1ac}
\end{table}

Ogni operatore ha acquisito indipendentemente le misurazioni lette sul minimetro. Per ciascuna misurazione ad assetto analogo ($F_{app}$ e fase di allungamento o accorciamento) si sono computati i valori di $\overline{\Delta x}$, $\sigma_{\Delta x}$, $\sigma_{\overline{\Delta x}}$ (Tabella \ref{tab:dati_grezzi_1ac}).
%Per la prima acquisizione dati si è scelto di calcolare separatamente due diversi valori di K, mantenendo separati i dati raccolti quando il filo veniva allungato e quando il filo veniva accorciato. Si è utilizzato il metodo del minimo $\chi^2$ al fine di calcolare il coefficiente elastico, sfruttando la relazione lineare che lega l'allungamento o l'accorciamento del filo con la forza applicata per metterlo in tensione. La relazione è la seguente e prende il nome di legge di Hooke:
%\begin{equation*}
%       \left | x_i - x_0  \right |  = K \cdot  \left | F_i - F_0 \right |
%\end{equation*}
Il calcolo di $\sigma_{\overline{\Delta x}}$ rivela la presenza di errori nulli, pertanto al fine di non sottostimare l'errore sulla singola misura si è valutata opportunamente $\sigma_{\overline{\Delta x}}$ in confronto a $\sigma_{Dist. uni.}$, errore dalla distribuzione uniforme, in modo da utilizzare la stima più appropriata. I parametri della distribuzione uniforme risultano essere: PTL$=\SI{10}{\micro\meter}$ e $\text{Coeff. aff.}=\num{10}$.\\
Il confronto fra $\sigma_{\overline{\Delta x}}$ e $\sigma_{Dist. uni.}$ è stato compiuto per tutte le misurazioni, non solo per quelle ad errore nullo dovuto alle misure identiche compiute da tutti e 3 gli operatori. Nel caso in cui $\sigma_{\overline{\Delta x}}$ è risultata minore di $\sigma_{Dist. uni.}$, si è impiegata quest'ultima nei successivi calcoli.\\%ABBIAMO FORSE SOTTOSTIMATO L'ERRORE DUNQUE CAMBIARE PARAMETRI DI PTL

In modo analogo, per il calcolo dell'incertezza sulla forza applicata si è usata la distribuzione triangolare considerando come $PTL=2 \cdot \SI{150}{\micro\meter}$ ovvero  il doppio dello spessore della tacca incisa e coefficienti di affidabilità 1.\\
È stato realizzato un grafico che rappresentasse le coppie $(\Delta F_{app}, \Delta x)$\footnote{Si noti che d'ora in avanti $\overline{\Delta x}$ verrà indicato con $\Delta x$, come in questo caso.} sulle quali è stata eseguita un'interpolazione con il metodo del $\chi^2$.
%Per l'utilizzo del $\chi^2$ si sono confrontati i due errori e si è scelto strategicamente, in quando minore rispetto all'errore su $\Delta x$, di trascurare l'errore sulla forza e di utilizzarla come grandezza sull'asse delle ascisse.\\
Si sono pertanto ottenuti i valori dei coefficienti angolari K e delle intercette con l'asse delle ordinate distinguendo i parametri delle misure in allungamento e in accorciamento ed è stato associato ad essi l'errore derivante dalla propagazione.\\
E' stata calcolata la compatibilità tra i due valori di K ottenuti e tra le due intercette. Si è fatto riferimento alle seguenti per valutare $\lambda$ e la sua bontà:
\begin{equation*}%Comp
    \label{eq:cases}
    \begin{cases}
    0<\lambda\leq 1, & \text{Ottima}\\
    1<\lambda\leq2, & \text{Discreta}\\
    2<\lambda\leq3, & \text{Pessima}\\
    3<\lambda, & \text{Non compatibile}\\
    \end{cases}
\end{equation*}
Successivamente gli errori sui coefficienti angolari e sulle intercette sono stati valutati nuovamente prendendo invece in considerazione $\sigma_{\Delta x, post}$. Analogamente a quanto fatto precedentemente, si è calcolata nuovamente la compatibilità.\\
Viene riportato in Figura \ref{fig:nos_estensimetro.png} il grafico dell'andamento delle misure con le relative interpolazioni.

\begin{figure}[h!]
    \centering
        \caption{Estensimetro 10 con interpolazioni}
        \label{fig:nos_estensimetro.png}
        \includegraphics[width=0.5\textwidth]{nos_estensimetro.png}
\end{figure}

Per dare infine una stima del coefficiente K si è eseguita la media ponderata fra $K_{all}$ e $K_{acc}$ precedentemente ottenuti mediante la Formula \ref{eq:media_pond} così come per le intercette. Vengono riportati in Tabella \ref{tab:parametri_fit_1ac} i dati ottenuti, confrontando l'utilizzo di $\sigma_{\Delta x}$ o di $\sigma_{\Delta x, post.}$.

\begin{table}[h!]
    \centering
    \makebox[\textwidth]{%
    \begin{tabular}{|cc|c|c|c|c|}
        \hline
        &&$K \pm \sigma_{K}$&	$\langle K \rangle \pm \sigma_{\langle K \rangle}$&	\multirow{2}{*}{$\lambda_{K_{all}, K_{acc.}}$}&	\multirow{2}{*}{$Err.\%$} \\
        &&$[\si{\micronnewton}]$&$[\si{\micronnewton}]$&&\\ \hline
        \multicolumn{1}{|c|}{\multirow{2}{*}{$\sigma_{Dist. uni.}$}}
        &{\cellcolor[rgb]{0.85,0.85,0.85}}All&	{\cellcolor[rgb]{0.85,0.85,0.85}}$55.306 \pm 0.008$&
        \multirow{2}{*}{$55.227\pm0.006$}&
        \multirow{2}{*}{13}&
        \multirow{2}{*}{0.01}\\    \cline{2-3}
        \multicolumn{1}{|c|}{}& Acc&	$55.160 \pm 0.008$ & & &\\  
        \hline
        \multicolumn{1}{|c|}{\multirow{2}{*}{$\sigma_{post}$}}
        &{\cellcolor[rgb]{0.85,0.85,0.85}}All&	{\cellcolor[rgb]{0.85,0.85,0.85}}$55.3 \pm 0.2$&
        \multirow{2}{*}{$55.20\pm0.03$}&
        \multirow{2}{*}{0.6}&
        \multirow{2}{*}{0.06}\\    \cline{2-3} 
        \multicolumn{1}{|c|}{}& Acc&	$55.17\pm0.09$ & & &\\  \hline
    \end{tabular}
    }
    \caption{Stime del parametro $b$ coeff. angolare del fit per il decimo estensimetro}
    \label{tab:parametri_fit_1ac}
\end{table}


\begin{table}[h!]
    \centering
    \makebox[\textwidth]{%
    \begin{tabular}{|cc|c|c|c|c|}
        \hline
        &&$Int \pm \sigma_{Int}$&	$\langle Int \rangle \pm \sigma_{\langle Int \rangle}$&	\multirow{2}{*}{$\lambda_{Int_{all}, Int_{acc.}}$}&	\multirow{2}{*}{$Err.\%$} \\
        &&$[\si{\micro\metre}]$&$[\si{\micro\meter}]$&&\\ \hline
        \multicolumn{1}{|c|}{\multirow{2}{*}{$\sigma_{Dist. uni.}$}}
        &{\cellcolor[rgb]{0.85,0.85,0.85}}All&	{\cellcolor[rgb]{0.85,0.85,0.85}}$2.7 \pm 0.2$&
        \multirow{2}{*}{$2.2\pm0.1$}&
        \multirow{2}{*}{3.0}&
        \multirow{2}{*}{6.2}\\    \cline{2-3}
        \multicolumn{1}{|c|}{}& Acc&	$1.9 \pm 0.2$ & & &\\  
        \hline
        \multicolumn{1}{|c|}{\multirow{2}{*}{$\sigma_{post}$}}
        &{\cellcolor[rgb]{0.85,0.85,0.85}}All&	{\cellcolor[rgb]{0.85,0.85,0.85}}$4 \pm 4$&
        \multirow{2}{*}{$2.3\pm0.8$}&
        \multirow{2}{*}{0.5}&
        \multirow{2}{*}{34.5}\\    \cline{2-3} 
        \multicolumn{1}{|c|}{}& Acc&	$2\pm2$ & & &\\  \hline
    \end{tabular}
    }
    \caption{Stime del parametro $a$ intercetta del fit per il decimo estensimetro}
    \label{tab:parametri_fit_1ac}
\end{table}

\subsubsection*{Secondo Metodo - Misure ripetute}
\begin{table}[h!]
    \centering
    \caption{Misure ripetute a $\SI{400}{gp}$ e $\SI{1000}{gp}$}
    \label{tab:misure_ripetute}
    \begin{tabular}{|cc|c|c|c||c|c|c|}
        \hline 
        & \# & Operatore 1&	Operatore 2&	Operatore 3&	$\overline{\Delta x}$&	$\sigma$&	$\sigma_{\overline{\Delta x}}$\\
        &   &$[\SI{e-6}{\meter}] $&$[\SI{e-6}{\meter}] $&$[\SI{e-6}{\meter}] $&$[\SI{e-6}{\meter}] $&$[\SI{e-6}{\meter}] $&$[\SI{e-6}{\meter}] $\\
        \hline
        
        \multicolumn{1}{|c|}{\multirow{9}{*}{\rotatebox[origin=c]{90}{\textbf{Misure 400 gp in acc.}}}}
        &{\cellcolor[rgb]{0.85,0.85,0.85}}1&	{\cellcolor[rgb]{0.85,0.85,0.85}}   439&	{\cellcolor[rgb]{0.85,0.85,0.85}}   439&	{\cellcolor[rgb]{0.85,0.85,0.85}}   438&	{\cellcolor[rgb]{0.85,0.85,0.85}}   438.7& {\cellcolor[rgb]{0.85,0.85,0.85}} 	0.6    &  {\cellcolor[rgb]{0.85,0.85,0.85}} 0.3\\
        \multicolumn{1}{|c|}{}&2&	435&	433&	435&	434.3&	1.1&	0.6\\
        \multicolumn{1}{|c|}{}&{\cellcolor[rgb]{0.85,0.85,0.85}}3&	{\cellcolor[rgb]{0.85,0.85,0.85}}   438&	{\cellcolor[rgb]{0.85,0.85,0.85}}   439.0&	{\cellcolor[rgb]{0.85,0.85,0.85}}   438&	{\cellcolor[rgb]{0.85,0.85,0.85}}   438.3&  {\cellcolor[rgb]{0.85,0.85,0.85}}	0.6&       {\cellcolor[rgb]{0.85,0.85,0.85}}  0.3\\
        \multicolumn{1}{|c|}{}&4&	439&	439&	438&	438.7&	0.6&	0.3\\
        \multicolumn{1}{|c|}{}&{    \cellcolor[rgb]{0.85,0.85,0.85}}5&	{\cellcolor[rgb]{0.85,0.85,0.85}}   441&	{\cellcolor[rgb]{0.85,0.85,0.85}}   441&	{\cellcolor[rgb]{0.85,0.85,0.85}}   435&	{\cellcolor[rgb]{0.85,0.85,0.85}}   439.0& {\cellcolor[rgb]{0.85,0.85,0.85}}	3.5&	{\cellcolor[rgb]{0.85,0.85,0.85}}2.0\\
        \multicolumn{1}{|c|}{}&6&	442&	442&	444&	442.7&	1.2&	0.7\\
        \multicolumn{1}{|c|}{}&{    \cellcolor[rgb]{0.85,0.85,0.85}}7& {\cellcolor[rgb]{0.85,0.85,0.85}}   431&	{\cellcolor[rgb]{0.85,0.85,0.85}}   431&	{\cellcolor[rgb]{0.85,0.85,0.85}}   432&	{\cellcolor[rgb]{0.85,0.85,0.85}}   431.3&  {\cellcolor[rgb]{0.85,0.85,0.85}}	0.6& 	{\cellcolor[rgb]{0.85,0.85,0.85}}   0.3\\
        \multicolumn{1}{|c|}{}&8&	432&	432&	432&	432.0&    0.0&	0.0\\
        \multicolumn{1}{|c|}{}& {\cellcolor[rgb]{0.85,0.85,0.85}}   9&	{\cellcolor[rgb]{0.85,0.85,0.85}}   435&	{\cellcolor[rgb]{0.85,0.85,0.85}}   434&	{\cellcolor[rgb]{0.85,0.85,0.85}}   434& {\cellcolor[rgb]{0.85,0.85,0.85}}   434.3&	{\cellcolor[rgb]{0.85,0.85,0.85}}   0.6&	{\cellcolor[rgb]{0.85,0.85,0.85}}   0.3\\  \hline \hline
        
        \multicolumn{1}{|c|}{\multirow{9}{*}{\rotatebox[origin=c]{90}{\textbf{Misure 400 gp in all.}}}}
        &{\cellcolor[rgb]{0.85,0.85,0.85}}1&	{\cellcolor[rgb]{0.85,0.85,0.85}}468&	{\cellcolor[rgb]{0.85,0.85,0.85}}459&	{\cellcolor[rgb]{0.85,0.85,0.85}}458&	{\cellcolor[rgb]{0.85,0.85,0.85}}461.7& {\cellcolor[rgb]{0.85,0.85,0.85}}5.5	&	{\cellcolor[rgb]{0.85,0.85,0.85}}3.2\\
        \multicolumn{1}{|c|}{}&2&	456&	453&	455&	454.7&	1.6&	0.9\\
        \multicolumn{1}{|c|}{}&{\cellcolor[rgb]{0.85,0.85,0.85}}3&	{\cellcolor[rgb]{0.85,0.85,0.85}}458&	{\cellcolor[rgb]{0.85,0.85,0.85}}458&	{\cellcolor[rgb]{0.85,0.85,0.85}}458&	{\cellcolor[rgb]{0.85,0.85,0.85}}458.0&   {\cellcolor[rgb]{0.85,0.85,0.85}}0.0	&	{\cellcolor[rgb]{0.85,0.85,0.85}}0.0\\
        \multicolumn{1}{|c|}{}&4&	459&	460&	460&	459.7&	0.6&	0.3\\
        \multicolumn{1}{|c|}{}&{\cellcolor[rgb]{0.85,0.85,0.85}}5&	{\cellcolor[rgb]{0.85,0.85,0.85}}457&	{\cellcolor[rgb]{0.85,0.85,0.85}}456&	{\cellcolor[rgb]{0.85,0.85,0.85}}455&	{\cellcolor[rgb]{0.85,0.85,0.85}}456.0&   {\cellcolor[rgb]{0.85,0.85,0.85}}1.0	&   {\cellcolor[rgb]{0.85,0.85,0.85}}0.6\\
        \multicolumn{1}{|c|}{}&6&	460&	460&	460&	460&	0.0&	0.0\\
        \multicolumn{1}{|c|}{}&{\cellcolor[rgb]{0.85,0.85,0.85}}7&	{\cellcolor[rgb]{0.85,0.85,0.85}}459&	{\cellcolor[rgb]{0.85,0.85,0.85}}459&	{\cellcolor[rgb]{0.85,0.85,0.85}}460&	{\cellcolor[rgb]{0.85,0.85,0.85}}459.3& {\cellcolor[rgb]{0.85,0.85,0.85}}0.6	&	{\cellcolor[rgb]{0.85,0.85,0.85}}0.3\\
        \multicolumn{1}{|c|}{}&8&	460&	460&	460&	460.0&	0.0&	0.0\\
        \multicolumn{1}{|c|}{}&{\cellcolor[rgb]{0.85,0.85,0.85}}9&	{\cellcolor[rgb]{0.85,0.85,0.85}}459&	{\cellcolor[rgb]{0.85,0.85,0.85}}459&	{\cellcolor[rgb]{0.85,0.85,0.85}}460&	{\cellcolor[rgb]{0.85,0.85,0.85}}459.3& {\cellcolor[rgb]{0.85,0.85,0.85}}0.6	&	{\cellcolor[rgb]{0.85,0.85,0.85}}0.3\\ \hline \hline
        
        \multicolumn{1}{|c|}{\multirow{8}{*}{\rotatebox[origin=c]{90}{\textbf{Misure 1000 gp in acc.}}}}
        &1& 1743&	1745&	1745&	1744.3& 	1.2&	0.7\\
        \multicolumn{1}{|c|}{}&{\cellcolor[rgb]{0.85,0.85,0.85}}2& {\cellcolor[rgb]{0.85,0.85,0.85}}1749&	{\cellcolor[rgb]{0.85,0.85,0.85}}1749&	{\cellcolor[rgb]{0.85,0.85,0.85}}1749&	{\cellcolor[rgb]{0.85,0.85,0.85}}1749.0& {\cellcolor[rgb]{0.85,0.85,0.85}}0.0	&	{\cellcolor[rgb]{0.85,0.85,0.85}}0.0\\
        \multicolumn{1}{|c|}{}&3& 1748&	1749&	1748&	1748.3& 	0.6&	0.3\\
        \multicolumn{1}{|c|}{}&{\cellcolor[rgb]{0.85,0.85,0.85}}4& {\cellcolor[rgb]{0.85,0.85,0.85}}1743&	{\cellcolor[rgb]{0.85,0.85,0.85}}1743&	{\cellcolor[rgb]{0.85,0.85,0.85}}1744&	{\cellcolor[rgb]{0.85,0.85,0.85}}1743.3& {\cellcolor[rgb]{0.85,0.85,0.85}}0.6	&	{\cellcolor[rgb]{0.85,0.85,0.85}}0.3\\
        \multicolumn{1}{|c|}{}&5& 1743&	1744&	1744&	1743.7& 	0.6&	0.3\\
        \multicolumn{1}{|c|}{}&{\cellcolor[rgb]{0.85,0.85,0.85}}6& {\cellcolor[rgb]{0.85,0.85,0.85}}1748&	{\cellcolor[rgb]{0.85,0.85,0.85}}1748&	{\cellcolor[rgb]{0.85,0.85,0.85}}1746&	{\cellcolor[rgb]{0.85,0.85,0.85}}1747.3& {\cellcolor[rgb]{0.85,0.85,0.85}}1.2&	{\cellcolor[rgb]{0.85,0.85,0.85}}0.7\\
        \multicolumn{1}{|c|}{}&7& 1750&	1750&	1749&	1749.7&	0.6&	0.3\\
        \multicolumn{1}{|c|}{}&{\cellcolor[rgb]{0.85,0.85,0.85}}8& {\cellcolor[rgb]{0.85,0.85,0.85}}1750&	{\cellcolor[rgb]{0.85,0.85,0.85}}1750&	{\cellcolor[rgb]{0.85,0.85,0.85}}1750&	{\cellcolor[rgb]{0.85,0.85,0.85}}1750.0& {\cellcolor[rgb]{0.85,0.85,0.85}}0.0	&	{\cellcolor[rgb]{0.85,0.85,0.85}}0.0\\ \hline \hline
        
        \multicolumn{1}{|c|}{\multirow{8}{*}{\rotatebox[origin=c]{90}{\textbf{Misure 1000 gp in all.}}}}
        &1& 1751&	1762&	1752&	1755.0& 	6.1&	3.5\\
        \multicolumn{1}{|c|}{}&{\cellcolor[rgb]{0.85,0.85,0.85}}2& {\cellcolor[rgb]{0.85,0.85,0.85}}1762&	{\cellcolor[rgb]{0.85,0.85,0.85}}1768&	{\cellcolor[rgb]{0.85,0.85,0.85}}1762&	{\cellcolor[rgb]{0.85,0.85,0.85}}1764.0& {\cellcolor[rgb]{0.85,0.85,0.85}}3.5	&	{\cellcolor[rgb]{0.85,0.85,0.85}}2.0\\
        \multicolumn{1}{|c|}{}&3& 1768&	1764&	1767&	1766.3& 2.1	&	1.2\\
        \multicolumn{1}{|c|}{}&{\cellcolor[rgb]{0.85,0.85,0.85}}4& {\cellcolor[rgb]{0.85,0.85,0.85}}1765&	{\cellcolor[rgb]{0.85,0.85,0.85}}1765&	{\cellcolor[rgb]{0.85,0.85,0.85}}1765&	{\cellcolor[rgb]{0.85,0.85,0.85}}1765.0& {\cellcolor[rgb]{0.85,0.85,0.85}}0.0	&	{\cellcolor[rgb]{0.85,0.85,0.85}}0.0\\
        \multicolumn{1}{|c|}{}&5& 1765&	1767&	1765&	1765.7&	1.2&	0.7\\
        \multicolumn{1}{|c|}{}&{\cellcolor[rgb]{0.85,0.85,0.85}}6& {\cellcolor[rgb]{0.85,0.85,0.85}}1764&	{\cellcolor[rgb]{0.85,0.85,0.85}}1763&	{\cellcolor[rgb]{0.85,0.85,0.85}}1766&	{\cellcolor[rgb]{0.85,0.85,0.85}}1764.3& {\cellcolor[rgb]{0.85,0.85,0.85}}11.5	&	{\cellcolor[rgb]{0.85,0.85,0.85}}0.9\\
        \multicolumn{1}{|c|}{}&7& 1762&	1762&	1762&	1762.0&	0.0&	0.0\\
        \multicolumn{1}{|c|}{}&{\cellcolor[rgb]{0.85,0.85,0.85}}8& {\cellcolor[rgb]{0.85,0.85,0.85}}1762&	{\cellcolor[rgb]{0.85,0.85,0.85}}1762&	{\cellcolor[rgb]{0.85,0.85,0.85}}1762&	{\cellcolor[rgb]{0.85,0.85,0.85}}1762.0& {\cellcolor[rgb]{0.85,0.85,0.85}}0.0	&	{\cellcolor[rgb]{0.85,0.85,0.85}}0.0\\ \hline
    \end{tabular}

\end{table}

\begin{table}[h!]
    \centering
    \begin{tabular}{|c|c|c|c|c|}
        \hline
          \multicolumn{2}{|c|}{\multirow{2}{*}{}} & ${\overline{\Delta x}}^\ast$& $\sigma$ & $\sigma_{{\overline{\Delta x}}^\ast}$\\
          \multicolumn{2}{|c|}{} & $[\SI{e-6}{\meter}]$ & $[\SI{e-6}{\meter}]$ & $[\SI{e-6}{\meter}]$\\ 
          \hline
          \multirow{2}{*}{\textbf{400 gp}}& {\cellcolor[rgb]{0.85,0.85,0.85}}Alto & {\cellcolor[rgb]{0.85,0.85,0.85}}436.6&	{\cellcolor[rgb]{0.85,0.85,0.85}}3.8&	{\cellcolor[rgb]{0.85,0.85,0.85}}1.3\\ \cline{2-5}
          &Basso&   458.7&	2.2&	0.7\\
          \hline
          \multirow{2}{*}{\textbf{1000 gp}}& {\cellcolor[rgb]{0.85,0.85,0.85}}Alto & {\cellcolor[rgb]{0.85,0.85,0.85}}1747.0&	{\cellcolor[rgb]{0.85,0.85,0.85}}2.8&	{\cellcolor[rgb]{0.85,0.85,0.85}}1.0\\ \cline{2-5}
          &Basso&   1763.0&	3.6&	1.3\\
         \hline
    \end{tabular}
    \caption{Medie e Dev. Std. Misure Ripetute}
    \label{tab:medie_misure_ripetute}
\end{table}{}

Per il calcolo di K tramite misure ripetute sono stati utilizzati due procedimenti differenti.\\
\paragraph{Primo procedimento}
Nel primo caso, al fine di apprezzare eventuali evidenze sperimentali celate dal metodo precedente, si è optato per il calcolo del coefficiente di elasticità direttamente mediante la legge qui riportata:\\% calcolando $\overline{\Delta x}$ per ogni campione di misurazioni.\\ 
\begin{equation*}
    K=\frac{\left | \Delta x \right |}{\left | \Delta F \right |}=\frac{\left | x_{i}-x_{0} \right |}{\left | F_{i, app}- F_{0, app} \right |}
\end{equation*}
In primo luogo si è proceduto con il calcolo delle medie delle misurazioni di ogni operatore, nello specifico per le misure di allungamento e accorciamento per $F_{letta}$ pari a $\SI{400}{gp}$ e $\SI{1000}{gp}$ aumentando o diminuendo la forza dal dinamometro di $\SI{200}{gp}$ dalla tacca prescelta (Tabella \ref{tab:misure_ripetute}). Di ogni campione di lunghezze $\{ {\Delta x}_i \}_{400, all}$, $\{ {\Delta x}_i \}_{400, acc}$, $\{ {\Delta x}_i \}_{1000, all}$ e $\{ {\Delta x}_i \}_{1000, acc}$ si è calcolata la media ${\overline{\Delta x}}^\ast$, associandovi come errore la relativa $\sigma_{\overline{\Delta x}^\ast}$ (Tabella \ref{tab:medie_misure_ripetute}).\\

Per ogni campione, utilizzando la legge, si sono ottenuti 4 valori differenti di coefficiente elastico ottenendo $K_{400, all}$, $K_{400, acc}$, $K_{1000, all}$ e $K_{1000, acc}$ impiegando come allungamento ${\overline{\Delta x}}^\ast$ e come $\Delta F$ il valore ottenuto dalla differenza tra $\SI{400}{gp}$ (oppure $\SI{1000}{gp} $) e $ \SI{200}{gp}$. L'errore di ogni K è stato calcolato utilizzando la propagazione degli errori tramite la seguente:
\begin{equation*}
\sigma_K \approx \sqrt{ \frac{\partial K }{\partial F_i} \Big|_{\ast}^2 \cdot  \sigma_{ F_i}^2 +
\frac{\partial K }{\partial F_0} \Big|_{\ast}^2\cdot  \sigma_{ F_0}^2 +
 \frac{\partial K }{\partial x_i}\Big|_{\ast}^2 \cdot  \sigma_{x_i}^2 +
 \frac{\partial K }{\partial x_0}\Big|_{\ast}^2\cdot  \sigma_{x_0} ^2 }
\end{equation*}

E' stata poi effettuata la media ponderata dei due coefficienti elastici in allungamento ($K_{400, all}$ e $K_{1000, all}$) e dei due in accorciamento ($K_{400, acc}$ e $K_{1000, acc}$) sempre associandovi il relativo errore e valutando la compatibilità fra ${\langle K \rangle}_{all}$ e ${\langle K \rangle}_{acc}$. Infine per restituire un valore unico $\langle K \rangle \pm \sigma_{\langle K \rangle}$ si è nuovamente calcolata la media ponderata fra questi due ultimi valori. Tutti i dati citati sono riportati in Tabella \ref{tab:2ac_1metodo}.\\


\begin{table}[h!]
\centering
\caption{Seconda acquisizione - Primo metodo}
\label{tab:2ac_1metodo}

    \begin{tabular}{|c|c|c|c||c|c|c|}
        \hline
        \multicolumn{2}{|c|}{}  & K &  \multirow{2}{*}{$\lambda_{400, 1000}$} & $K_{pond_{400, 1000}}$ & \multirow{2}{*}{$\lambda_{alto, basso}$}  & $K_{pond_{alto, basso}}$ \\
        \multicolumn{2}{|c|}{}  & [$\si{\micronnewton}]$ && [$\si{\micronnewton}$] && [$\si{\micronnewton}]$ \\\hline
        \multirow{2}{*}{Basso} & {\cellcolor[rgb]{0.85,0.85,0.85}}400  & {\cellcolor[rgb]{0.85,0.85,0.85}}$58.5\pm0.3$  & \multirow{2}{*}{6.9}   & \multirow{2}{*}{$56.32\pm0.08$} & \multirow{4}{*}{6.0} & \multirow{4}{*}{$55.99\pm0.05$} \\\cline{2-3}
        & 1000 & $56.19\pm0.08$ &  &  &  & \\\cline{1-5}
        \multirow{2}{*}{Alto}  & {\cellcolor[rgb]{0.85,0.85,0.85}}400  & {\cellcolor[rgb]{0.85,0.85,0.85}}$55.5\pm0.3$  & \multirow{2}{*}{0.5} & \multirow{2}{*}{$55.68\pm0.08$} & & \\\cline{2-3}
        & 1000 & $55.69\pm0.08$ &  &  &  &\\ \hline 
    \end{tabular}
\end{table}



\paragraph{Secondo procedimento}
Il secondo procedimento utilizzato per il calcolo di K tramite le misure ripetute è consistito nel calcolo del $\chi^2$ su tutte le misure ottenute a $\SI{400}{gp}$ e $\SI{1000}{gp}$ con l'accortezza di distinguere i dati presi in allungamento ed in accorciamento. Operando secondo questo schema si sono ottenute due rette di interpolazione, una per i dati delle misure "in allungamento" ed una per quelli "in accorciamento". I due coefficienti angolari rappresentano i coefficienti rispettivamente di allungamento e di accorciamento e da essi si è calcolata la compatibilità fra ${K}_{acc}$ e ${K}_{all}$ e ricavata la media ponderata, rappresentativa della costante elastica dell'estensimetro in esame. Vengono riportati i dati qui citati in Tabella \ref{tab:2ac_2met}. %I PARAMETRI DEL FIT

\begin{table}[h!]
\centering
\caption{Seconda acquisizione - Secondo Metodo}
\label{tab:2ac_2met}
\begin{tabular}{|c|c|c|c|c|c|c|}
    \hline
    & $K_{\chi^2}$ & \multirow{2}{*}{$\lambda$} & $K_{pond}$ & $Intercetta_{\chi^2}$ & \multirow{2}{*}{$\lambda$} & $Intercetta_{pond}$\\
    &$[\si{\micronnewton}]$&&$[\si{\micronnewton}]$& $[\si{\micro\meter}]$ & & $[\si{\micro\meter}]$\\ \hline
    {\cellcolor[rgb]{0.85,0.85,0.85}}Basso & {\cellcolor[rgb]{0.85,0.85,0.85}}$55.42\pm0.06$  & \multirow{2}{*}{2.8}  & \multirow{2}{*}{$55.53\pm0.05$} & {\cellcolor[rgb]{0.85,0.85,0.85}}$24\pm1$   &   \multirow{2}{*}{12} & \multirow{2}{*}{$13.44\pm0.05$}   \\ \cline{1-2} \cline{5-5}
    Alto  & $55.68\pm0.07$ & & & $-0.2\pm1.5$ & & \\ \hline
\end{tabular}
\end{table}

% METODO DE SALVADOR
\subsubsection*{Errore sistematico}\label{par:errore_desalva}
Per la valutazione dell'errore sistematico del quale sono affette tutte le misurazioni effettuate è stato utilizzato un metodo che valuta i dati ottenuti dalle misurazioni ripetute. Si è considerato errore sistematico la differenza fra $K_{all}$ e $K_{acc}$ e per valutarla si è impiegata la seguente:


\begin{equation*}
    \Delta_{K} = K_{acc}- K_{all}=\frac{x^{Acc}_{1000}-x^{Acc}_{400}}{\Delta F}-\frac{x^{All}_{1000}-x^{All}_{400}}{\Delta F} \\
\end{equation*}
    
\makebox[\textwidth][r]{
\newline
\begin{equation*}
    \sigma_{\Delta_K}\approx \sqrt{
    \frac{\partial \Delta_K }{\partial x^{Acc}_{1000}} \Big|_{\ast}^2\cdot  \sigma_{x^{Acc}_{1000}} ^2+
     \frac{\partial \Delta_K}{\partial x^{Acc}_{400}}\Big|_{\ast}^2\cdot  \sigma_{x^{Acc}_{400}} ^2+
     \frac{\partial \Delta_K }{\partial x^{All}_{1000}} \Big|_{\ast}^2\cdot  \sigma_{x^{All}_{1000}} ^2+
     \frac{\partial \Delta_K }{\partial x^{All}_{400}} \Big|_{\ast}^2\cdot  \sigma_{x^{All}_{400}} ^2+
     \frac{\partial \Delta_K}{\partial F_i} \Big|_{\ast}^2\cdot  \sigma_{F_i} ^2+
     \frac{\partial \Delta_K}{\partial F_0} \Big|_{\ast}^2\cdot  \sigma_{F_0} ^2}
\end{equation*}
}
\\
\\
In cui $x_{1000 / 400}^{all / acc}$ corrisponde a $\overline{\Delta x}^\ast$, $\sigma_{x}$ è $\sigma_{\overline{\Delta x}^\ast}$ e $\Delta F$ la variazione della forza tra $\SI{1000}{gp}$ e $\SI{400}{gp}$

I calcoli restituiscono un errore sistematico di $\Delta_{K} = (-0.3 \pm 0.2) \si{\micronnewton}$. 

\begin{figure}[h!]
    \centering
    \includegraphics[width=0.6\textwidth]{de_salvador.png}
    \caption{Errore Sistematico}
    \label{fig:de_salvador}
\end{figure}

\subsubsection*{Definizione miglior protocollo di misura}
I metodi descritti precedentemente sono necessari per definire il miglior metodo di misura al fine di rendere le misurazioni più accurate possibili.
Si rimanda alla Discussione, \ref{sec:discussione_prima_parte}\S \textit{ Definizione miglior protocollo misura}. 

%TABELLA CONFRONTO DUE METODI PER PROTOCOLLO MISURA E ALCUNI RIGHE PER DIRE COSA CE IN TABELLA

\newpage
\subsection{Analisi sistematica di tutti gli estensimetri}
\subsubsection*{Stime di $K$}
Dai dati forniti, si sono calcolati i $\Delta x$ come differenza in modulo tra le misure $x_i$ e $x_0$ separatamente per allungamento e accorciamento, corrispondenti ad una $\Delta F_{app} =\left | F_{i, app} - F_{0, app} \right |$. A ciascun valore di x si è associato un errore derivante dalla distribuzione uniforme con PTL pari a $\SI{5}{\micro\meter}$ e coefficiente di affidabilità pari a 1, e conseguentemente tramite la propagazione degli errori casuali si è calcolato l'errore di  $\Delta x$. L'errore associato a $\Delta F$ è stato derivato dalla distribuzione triangolare con i parametri descritti precedentemente.\\
Similmente a quanto descritto nella Sezione \ref{sec:prima_parte}, dopo una rappresentazione grafica delle coppie $(\Delta F, \Delta x)$, si sono calcolati i $K_{all}$ e i $K_{acc}$ e le intercette per tutti gli estensimetri dotati di filo in metallo tramite il metodo del $\chi^2$. I relativi errori sono stati calcolati tramite la formula di propagazione degli errori tramite la $\sigma_{post}$.\\
Si è poi calcolata la compatibilità fra $K_{all}$ e $K_{acc}$ e fra $Intercetta_{all}$ e $Intercetta_{acc}$ e la media ponderata tra i $K_{all}$ e $K_{acc}$ come stima del ${\langle K \rangle }\pm \sigma_{{\langle K\rangle }}$ per ciascun estensimetro come viene esposto nella Tabella \ref{tab:k_est}. Non si riportano i valori del decimo estensimetro in quanto approfonditamente analizzato nella prima parte. 

\begin{figure}[H]
    \centering
    \subfloat[Estensimetro 1]{
        \label{fig:1_estensimetro}
        \includegraphics[width=0.5\textwidth]{1_estensimetro.png}
    }
    \subfloat[Estensimetro 2]{
        \label{fig:2_estensimetro}
        \includegraphics[width=0.5\textwidth]{2_estensimetro.png}
    }
    \newline
    \subfloat[Estensimetro 3]{
        \label{fig:3_estensimetro}
        \includegraphics[width=0.5\textwidth]{3_estensimetro.png}
    }
    \subfloat[Estensimetro 4]{
        \label{fig:4_estensimetro}
        \includegraphics[width=0.5\textwidth]{4_estensimetro.png}
    }
    \newline
    \subfloat[Estensimetro 5]{
        \label{fig:5_estensimetro}
        \includegraphics[width=0.5\textwidth]{5_estensimetro.png}
    }
    \subfloat[Estensimetro 6]{
        \label{fig:6_estensimetro}
        \includegraphics[width=0.5\textwidth]{6_estensimetro.png}
    }
    \newline
    \subfloat[Estensimetro 7]{
        \label{fig:7_estensimetro}
        \includegraphics[width=0.5\textwidth]{7_estensimetro.png}
    }
    \subfloat[Estensimetro 8]{
        \label{fig:8_estensimetro}
        \includegraphics[width=0.5\textwidth]{8_estensimetro.png}
    }
\end{figure}

\begin{figure}[h!]
    \centering
    
    \subfloat[Estensimetro 9]{
        \label{fig:9_estensimetro}
        \includegraphics[width=0.5\textwidth]{9_estensimetro.png}
    }
\end{figure}


\begin{figure}[h!]
\subfloat[Estensimetro 11, tungsteno]{
        \label{fig:11_estensimetro}
        \includegraphics[width=0.5\textwidth]{tungsteno.png}
    }
    \subfloat[Estensimetro 12, ottone]{
        \label{fig:12_estensimetro}
        \includegraphics[width=0.5\textwidth]{ottone.png}
    }
    \caption{Grafici Estensimetri}
\end{figure}

\begin{table}[h!]
    \caption{K e intercette estensimetri}
    \label{tab:k_est}
    \begin{center}
    \makebox[\textwidth]{
    \begin{tabular}{|c|c|c|c|c||c|c|c|c|}
    \hline
    \multirow{2}{*}{\textbf{Estensimetro}} & $K_{all}$ & $K_{acc}$  & $\langle K\rangle$ & \multirow{2}{*}{$\lambda_K$} & $Int_{all}$ & $Int_{acc}$ & $\langle Int\rangle$ & \multirow{2}{*}{$\lambda_I$}\\
    & $[\si{\micro\metre\per\newton}]$ & $[\si{\micro\metre\per\newton}]$ & $[\si{\micro\metre\per\newton}]$ & & $[\si{\micro\metre}]$& $[\si{\micro\metre}]$& $[\si{\micro\metre}]$&\\ \hline
    {\cellcolor[rgb]{0.85,0.85,0.85}}1 & {\cellcolor[rgb]{0.85,0.85,0.85}}$59\pm2$ & {\cellcolor[rgb]{0.85,0.85,0.85}}$58\pm2$ & {\cellcolor[rgb]{0.85,0.85,0.85}}$58\pm2$ & {\cellcolor[rgb]{0.85,0.85,0.85}}$0.2$ & {\cellcolor[rgb]{0.85,0.85,0.85}}$26\pm51$ & {\cellcolor[rgb]{0.85,0.85,0.85}}$7\pm50$ & {\cellcolor[rgb]{0.85,0.85,0.85}}$16\pm36$ & {\cellcolor[rgb]{0.85,0.85,0.85}}$0.3$\\ \hline
    2 & $50\pm3$ & $50\pm3$ & $50\pm2$ & $0.03$ & $25\pm62$ & $19\pm62$ & $22\pm44$ & $0.1$\\ \hline
    {\cellcolor[rgb]{0.85,0.85,0.85}}3 & {\cellcolor[rgb]{0.85,0.85,0.85}}$48\pm2$ & {\cellcolor[rgb]{0.85,0.85,0.85}}$48\pm1$ & {\cellcolor[rgb]{0.85,0.85,0.85}}$48\pm1$ & {\cellcolor[rgb]{0.85,0.85,0.85}}$0.2$ & {\cellcolor[rgb]{0.85,0.85,0.85}}$-5\pm30$ & {\cellcolor[rgb]{0.85,0.85,0.85}}$-19\pm35$ & {\cellcolor[rgb]{0.85,0.85,0.85}}$-11\pm23$ & {\cellcolor[rgb]{0.85,0.85,0.85}}$0.3$\\ \hline
    4 & $70\pm3$ & $71\pm3$ & $71\pm2$ & $0.2$ & $-57\pm53$ & $-13\pm58$ & $-38\pm39$ & $0.6$\\ \hline
    {\cellcolor[rgb]{0.85,0.85,0.85}}5 & {\cellcolor[rgb]{0.85,0.85,0.85}}$29\pm3$ & {\cellcolor[rgb]{0.85,0.85,0.85}}$28\pm3$ & {\cellcolor[rgb]{0.85,0.85,0.85}}$28\pm2$ & {\cellcolor[rgb]{0.85,0.85,0.85}}$0.2$ & {\cellcolor[rgb]{0.85,0.85,0.85}}$69\pm55$ & {\cellcolor[rgb]{0.85,0.85,0.85}}$77\pm59$ & {\cellcolor[rgb]{0.85,0.85,0.85}}$73\pm40$ & {\cellcolor[rgb]{0.85,0.85,0.85}}$0.1$\\ \hline
    6 & $37\pm2$ & $37\pm2$ & $37\pm1$ & $0.1$ & $-6\pm42$ & $11\pm42$ & $3\pm30$ & $0.3$\\ \hline
    {\cellcolor[rgb]{0.85,0.85,0.85}}7 & {\cellcolor[rgb]{0.85,0.85,0.85}}$78\pm2$ & {\cellcolor[rgb]{0.85,0.85,0.85}}$78\pm2$ & {\cellcolor[rgb]{0.85,0.85,0.85}}$78\pm1$ & {\cellcolor[rgb]{0.85,0.85,0.85}}$0.02$ & {\cellcolor[rgb]{0.85,0.85,0.85}}$-66\pm43$ & {\cellcolor[rgb]{0.85,0.85,0.85}}$-53\pm41$ & {\cellcolor[rgb]{0.85,0.85,0.85}}$-60\pm30$ & {\cellcolor[rgb]{0.85,0.85,0.85}}$0.2$\\ \hline
    8 & $53\pm2$ & $52\pm2$ & $53\pm1$ & $0.2$ & $-21\pm47$ & $-16\pm40$ & $-18\pm31$ & $0.1$\\ \hline
    {\cellcolor[rgb]{0.85,0.85,0.85}}9 & {\cellcolor[rgb]{0.85,0.85,0.85}}$70\pm3$ & {\cellcolor[rgb]{0.85,0.85,0.85}}$71\pm3$ & {\cellcolor[rgb]{0.85,0.85,0.85}}$70\pm3$ & {\cellcolor[rgb]{0.85,0.85,0.85}}$0.3$ & {\cellcolor[rgb]{0.85,0.85,0.85}}$-32\pm60$ & {\cellcolor[rgb]{0.85,0.85,0.85}}$4\pm62$ & {\cellcolor[rgb]{0.85,0.85,0.85}}$-14\pm43$ & {\cellcolor[rgb]{0.85,0.85,0.85}}$0.4$\\ \hline \hline
    Tungsteno & $62\pm4$ & $61\pm4$ & $61\pm3$ & $0.3$ & $-94\pm79$ & $0.2\pm75.5$ & $-45\pm55$ & $0.9$\\ \hline
    {\cellcolor[rgb]{0.85,0.85,0.85}}Ottone & {\cellcolor[rgb]{0.85,0.85,0.85}}$58\pm3$ & {\cellcolor[rgb]{0.85,0.85,0.85}}$58\pm3$ & {\cellcolor[rgb]{0.85,0.85,0.85}}$58\pm2$ & {\cellcolor[rgb]{0.85,0.85,0.85}}$0.1$ & {\cellcolor[rgb]{0.85,0.85,0.85}}$-75\pm68$ & {\cellcolor[rgb]{0.85,0.85,0.85}}$-85\pm69$ & {\cellcolor[rgb]{0.85,0.85,0.85}}$-80\pm48$ & {\cellcolor[rgb]{0.85,0.85,0.85}}$0.1$\\ \hline
    \end{tabular}}
    \end{center}
\end{table}
\newpage
\subsubsection*{Verifica della validità della legge di E}
Tramite l'analisi incrociata di diversi metodi è stata verificata la seguente legge utilizzata per il calcolo di E.
\begin{equation}
    \label{equation:legge_e}
    E=\frac{x_0}{S K}=\frac{4 x_0}{\pi D^{2}K}
\end{equation}
I metodi impiegati consistono nel verificare la proporzionalità dei parametri presenti nell'equazione sfruttandone la dipendenza lineare. Le varie verifiche della legge che sono state eseguite sono le seguenti:
\begin{enumerate}
    \item verificare la diretta proporzionalità tra $\langle K \rangle$ e la lunghezza del filo a riposo $x_0$, di tutti gli estensimetri aventi stessa sezione $S$.
    \item verificare la diretta proporzionalità tra $\langle K \rangle$ e il rapporto della sezione $1/S$ di tutti gli estensimetri aventi stessa lunghezza a riposo $x_{0}$.
    \item valutare la regolarità del rapporto $R=\frac{\langle K \rangle}{x_{0}}$, per tutti gli estensimetri con la stessa $S$, indipendentemente da $x_{0}$
    \item valutare la regolarità del prodotto $P={ \langle K \rangle} D^2 $, per tutti gli estensimetri aventi la stessa $x_{0}$, indipendentemente da $S$
\end{enumerate}
\paragraph{Primo Metodo}
Per la verifica della legge tramite il primo metodo (Grafico \ref{fig:a_sezione_cost}) si sono scelti i 4  estensimetri aventi la medesima sezione, e dunque lo stesso diametro, pari a \SI{279}{\micro\meter}. Tramite il calcolo dei $\langle K \rangle$ dei vari estensimetri selezionati, precedentemente effettuato, ed avendo a disposizione la lunghezza a riposo di ciascun filo, è stato elaborato un grafico avente sull'asse delle ascisse la lunghezza a riposo espressa in $\si{\milli\meter}$ e sull'asse delle ordinate il valore di K in $\si{\micronnewton}$, associandone adeguatamente l'errore che era stato già calcolato nelle sezioni precedenti.

\begin{figure}[h!]
    \centering
    \caption{Proporzionalità K e $x_{0}$}
    \subfloat[Verifica diretta proporzionalità tra K e $x_0$]{
        \label{fig:a_sezione_cost}
        \includegraphics[width=8cm]{a_sez_cost.png}
    }
    \subfloat[Tabella]{
        \begin{tabular}{|c|c|c|}
                \hline
                \textbf{L} & \textbf{K} & \textbf{Coeff. Ango}\\
                \si{[mm]}& $[\si{\micronnewton}]$ &$10^{-5}[\si{1/\newton}]$ \\
                \hline
                {\cellcolor[rgb]{0.85,0.85,0.85}}$600$& {\cellcolor[rgb]{0.85,0.85,0.85}}	$53\pm1$&   $7.8\pm0.5$\\    \cline{3-3}
                $700$&	$55.2\pm0.1$&    \textbf{Intercetta}\\\
                {\cellcolor[rgb]{0.85,0.85,0.85}}$800$&{\cellcolor[rgb]{0.85,0.85,0.85}}	$70\pm2$& $[\si{\micronnewton}]$\\    \cline{3-3}
                $950$&	$78\pm1$& $ 4\pm4$\\ 
                \hline
        \end{tabular}
        \label{tab:sez_cost}
    }
\end{figure}

\paragraph{secondo Metodo}
Per la verifica del secondo metodo (Grafico \ref{fig:a_lunghezza_cost}) sono stati selezionati invece gli estensimetri aventi lunghezza del filo d'acciaio pari a $\SI{950}{\milli\meter}$. E' stato realizzato un grafico avente sull'asse delle ascisse il valore di 1/S  calcolato per ogni estensimetro e sull'asse delle ordinate il valore di $\langle K \rangle$ dello stesso, associandone il relativo errore precedentemente calcolato. Si è inoltre determinata una retta interpolante i dati rappresentata nel grafico con una retta tratteggiata di colore blu chiaro.\\
A causa dell'incongruenza tra il valore rappresentato in blu sul grafico e gli altri dati, si è deciso di eseguire una seconda analisi più accurata, questa volta eliminando il dato considerato e determinando una nuova retta interpolante, rappresentata in rosso nel grafico. L'eliminazione è stata effettuata calcolando le frequenze di ottenere un dato valore in una regione ben delineata di grafico, ovvero tra la nuova retta interpolante e una parallela, ottenuta aumentando l'intercetta dell'interpolante di un valore arbitrariamente scelto. Anche tenendo conto dell'errore del quale risulta affetto il dato eliminato, si è riscontrato che la probabilità di ottenere lo stesso dato nella regione prima determinata è molto inferiore rispetto a quella calcolata per gli altri dati.
%bisogna specificare nella discussione tutto quello che è stato fatto, occhio alla distinzione tra analisi e discussisione 


\begin{figure}[h!]
    \centering
    %\makebox[\textwidth]{%
    \caption{Proporzionalità K e 1/S}
    \subfloat[Verifica diretta proporzionalità tra K e $1/S$]{
        \label{fig:a_lunghezza_cost}
        \includegraphics[width=8cm]{a_lung_cost .png}
    }
    \subfloat[Tabella]{
        \begin{tabular}{|c|c|c|}
                \hline
                \textbf{Rec Sez} & \textbf{K} & \textbf{Coeff. Ango}\\
                $10^{-5}[\si{1/\micro\meter}]$ & $[\si{\micronnewton}]$& $10^{6}[\si{\micro\metre\squared\per\newton}]$\\
                \hline
                {\cellcolor[rgb]{0.85,0.85,0.85}}$1.17$&{\cellcolor[rgb]{0.85,0.85,0.85}}	$58\pm2$& \multirow{1}{*}{$3.0\pm0.2$}\\
                $1.00$&	$50\pm2$&  \multirow{1}{*}{Dopo reiez. $4.7\pm0.2$}  \\
                {\cellcolor[rgb]{0.85,0.85,0.85}}$0.88$&{\cellcolor[rgb]{0.85,0.85,0.85}}	$48\pm1$&     \\    \cline{3-3}
                $1.37$&	$71\pm2$& \textbf{Intercetta}\\
                {\cellcolor[rgb]{1,0.5,0.5}}$1.41$&{\cellcolor[rgb]{0.85,0.85,0.85}}	{\cellcolor[rgb]{1,0.5,0.5}}$28\pm2$& $[\si{\micronnewton}]$\\\cline{3-3}
                $0.80$&   $37\pm1$&    \multirow{1}{*}{$17\pm2$}\\
                {\cellcolor[rgb]{0.85,0.85,0.85}}$1.64$&{\cellcolor[rgb]{0.85,0.85,0.85}}	$78\pm1$& Dopo reiez. $3\pm2$   \\
                \hline
        \end{tabular}
        \label{tab:lungh_cost}
    }
\end{figure}



\paragraph{Terzo Metodo}
Per quanto riguarda il calcolo del rapporto R, come per i metodi precedenti, si sono selezionati solo alcuni estensimetri,ovvero quelli aventi la stessa sezione ed è stato eseguito il calcolo per ognuno di essi del rapporto R tramite la seguente formula: $R=\frac{\langle K \rangle}{x_{0}}$. Ad esso è stato associato un errore calcolato tramite la propagazione degli errori, ovvero 
\begin{equation*}
    \sigma_R \approx \sqrt{ \frac{\partial R }{\partial K}\Big|_{\ast}^2 \cdot  \sigma_{K}^2 +  \frac{\partial R }{\partial x_0}\Big|_{\ast}^2 \cdot \sigma_{x_0}^2 }
\end{equation*}

In seguito è stato generato un grafico contenete i valori calcolati, nello specifico sull'asse delle ascisse sono stati posizionati i valori delle lunghezze a riposo degli estensimetri presi in considerazione e sull'asse delle ordinate il valore di R calcolato. E' stata poi determinata una retta interpolante i 4 dati rappresentati sul grafico.


\begin{figure}[h!]
    \centering
        \label{fig:rapporto}
        \caption{Rapporto}
    \subfloat[Verifica del rapporto R]{
            \label{fig:grafico_rapporto}
            \includegraphics[width=8cm]{rapporto.png}
        }
    \subfloat[Tabella]{
        \begin{tabular}{|c|c|c|}
                \hline
                \textbf{L} & \textbf{Rapporto} &\textbf{CoeffAngo}\\
                \si{[mm]} & ${10^{-5}} [\si{1/\newton}]$ & $10^{-12}[\si{1/{\newton \micro\meter}}]$\\
                \hline
                {\cellcolor[rgb]{0.85,0.85,0.85}}$600000$&	{\cellcolor[rgb]{0.85,0.85,0.85}}$8.8\pm0.2$&    $-8\pm6$\\  \cline{3-3}
                $700000$&	$7.88\pm0.03$&\textbf{Intercetta}\\
                {\cellcolor[rgb]{0.85,0.85,0.85}}$800000$&	{\cellcolor[rgb]{0.85,0.85,0.85}}$8.8\pm0.3$&    $10^{-5}[\si{1/\newton}]$\\   \cline{3-3}
                $950000$&	$8.2\pm0.1$&    $9.0\pm0.4$\\
                \hline
        \end{tabular}
        \label{tab:rapporto}
    }
\end{figure}

\paragraph{Quarto Metodo}
Infine per il quarto metodo, consistente nel calcolo del prodotto P, sono stati selezionati gli estensimetri aventi la stessa lunghezza a riposo ed è stato calcolato per ognuno di essi il quadrato del diametro. Conseguentemente è stato ricavato P tramite la seguente: $P= K \cdot D^2 $, e ad esso è stato attribuito l'errore ricavato dalla propagazione degli errori, ovvero 

\begin{equation*}
    \sigma_P \approx \sqrt{\frac{\partial P }{\partial K} \Big|_{\ast}^2 \cdot 
    \sigma_K^2 +  \frac{\partial R }{\partial D}\Big|_{\ast}^2 \cdot \sigma_D^2 }
\end{equation*}

E' stata conseguentemente determinata una retta interpolante le coppie di dati ed è stata rappresentata nel grafico in blu tratteggiato.\\
Poiché si dubitava della veridicità del dato rappresentato in blu nel grafico, così come già eseguito per la diretta proporzionalità tra K e 1/S, tramite il medesimo ragionamento è stata eseguita la reiezione di quest'ultimo ed è stata determinata una nuova retta interpolante i dati ottenuti, rappresentata in rosso nel grafico sottostante.

\begin{figure}[h!]
    \centering
        \label{fig:prodotto}
        \caption{Prodotto}
        \subfloat[Verifica del prodotto P]{
            \label{fig:grafico_prodotto}
            \includegraphics[width=8cm]{prodotto.png}
        }
        \subfloat[Tabella]{
            \begin{tabular}{|c|c|c|}
                \hline
                \textbf{D}^{2} & \textbf{Prodotto} &\textbf{CoeffAngo}\\
                $[\si{\micro\meter}^2]$ & $10^{6}[\si{\micro\meter\cubed\per\newton}]$ & $10^{6}[\si{\micro\meter\per\newton}]$\\
                \hline
                {\cellcolor[rgb]{0.85,0.85,0.85}}$77841$&	{\cellcolor[rgb]{0.85,0.85,0.85}}$6.1\pm0.2$&\multirow{1}{*}{$17\pm3$}\\
                {\cellcolor[rgb]{1,0.5,0.5}}$90000$&	{\cellcolor[rgb]{1,0.5,0.5}}$2.5\pm0.2$& Dopo reiez. $1\pm2$\\
                {\cellcolor[rgb]{0.85,0.85,0.85}}$93025$&	{\cellcolor[rgb]{0.85,0.85,0.85}}$6.6\pm0.2$& \\  \cline{3-3}
                $108900$&	$6.4\pm0.2$&\textbf{Intercetta}\\
                {\cellcolor[rgb]{0.85,0.85,0.85}}$126736$&	{\cellcolor[rgb]{0.85,0.85,0.85}}$6.3\pm0.3$&  $[\si{\micro\meter\cubed\per\newton}]$\\  \cline{3-3}
                $145161$&	$7.0\pm0.2$& \multirow{1}{*}{$3.0\pm0.2$} \\
                {\cellcolor[rgb]{0.85,0.85,0.85}}$160000$&	{\cellcolor[rgb]{0.85,0.85,0.85}}$5.9\pm0.3$& Dopo reiez. $6.3\pm0.2$\\
                
                \hline
            \end{tabular}
            \label{tab:prodott}
        }
\end{figure}

\subsubsection*{Stime di $E$}
Per la stima del modulo di Young sono stati impiegati e confrontati 3 metodi differenti.

\paragraph{Primo Metodo}
Per ciascun estensimetro si è calcolata la media ponderata $\langle E \rangle $ fra $E_{all}$ e $E_{acc}$, ciascuno di questi ultimi calcolato tramite la formula \ref{equation:legge_e} del paragrafo precedente. Gli errori su $\langle E \rangle$ sono stati propagati con la seguente:
\begin{equation}
    \label{eq:propagazione_particolare}
    \sigma_{E_{acc, all}} \approx E\sqrt{\frac{\sigma_{x_{0}}}{x_{0}}\Big|_{\ast}^2+\frac{\sigma_{K}}{K}\Big|_{\ast}^2+4\frac{\sigma_{D}}{D}\Big|_{\ast}^2}
\end{equation}
L'errore associato alla $\langle E \rangle $ invece è stato calcolato come errore della media ponderata.

\paragraph{Secondo Metodo}
Nel secondo metodo, per ciascun estensimetro, è stato calcolato $\langle K \rangle $ come media ponderata di $K_{all}$ e $K_{acc}$ e a partire da esso è stato ricavato $E_{pes}$ e l'errore $\sigma_{E_{pes}}$.

\paragraph{Terzo Metodo}
Nell'ultimo metodo si sono usate le stime di $\overline{K_{all}} \pm \sigma_{\overline{K}}$ e $\overline{K_{acc}} \pm \sigma_{\overline{K}}$ derivanti dalle medie e dalle deviazioni standard sulla media di campioni $\{ K_{all} \}_i$ e $\{ K_{acc}\}_i $ ottenuti da misure di $\Delta x$ non consecutive, così da risultare statisticamente indipendenti. Analogamente al primo metodo si sono calcolati $E_{all}$ e $E_{acc}$ dei quali si è infine ricavata la media ponderata $\overline{E}$.

Vengono riportati i valori ottenuti per gli estensimetri in acciaio nella Tabella \ref{tab:valori_e}.\\


\begin{table}[h!]
\centering
\caption{Valori di E ottenuti attraverso i diversi metodi}
\label{tab:valori_e}
    \begin{tabular}{|c|c|c|c|}
    \hline
    \multirow{2}{*}{\textbf{Estensimetro}} & \textbf{Primo metodo}  & \textbf{Secondo metodo} & \textbf{Terzo metodo}\\
    & $\langle E \rangle$ $[\si{\newton\per{\micro\metre}^2}]$ & $E_{pes}$ $[\si{\newton\per{\micro\metre}^2}]$ & $\overline{E}$ $[\si{\newton\per{\micro\metre}^2}]$\\\hline
    {\cellcolor[rgb]{0.85,0.85,0.85}}1 & {\cellcolor[rgb]{0.85,0.85,0.85}}$0.190 \pm 0.008$ & {\cellcolor[rgb]{0.85,0.85,0.85}}$0.190 \pm 0.007$ & {\cellcolor[rgb]{0.85,0.85,0.85}}$0.20 \pm 0.02$ \\ \hline
    2 & $0.19 \pm 0.01$ & $0.192 \pm 0.009$ & $0.24 \pm 0.04$ \\ \hline
    {\cellcolor[rgb]{0.85,0.85,0.85}}3 & {\cellcolor[rgb]{0.85,0.85,0.85}}$0.173 \pm 0.006$ & {\cellcolor[rgb]{0.85,0.85,0.85}}$0.173 \pm 0.005$ & {\cellcolor[rgb]{0.85,0.85,0.85}}$0.19 \pm 0.03$ \\ \hline
    4 & $0.184 \pm 0.007$ & $0.184 \pm 0.006$ & $0.23 \pm 0.03$ \\ \hline
    {\cellcolor[rgb]{1,0.5,0.5}}5& {\cellcolor[rgb]{1,0.5,0.5}}$0.48 \pm 0.05$ & {\cellcolor[rgb]{1,0.5,0.5}}$0.48 \pm 0.03$ & {\cellcolor[rgb]{1,0.5,0.5}}$0.50 \pm 0.08$ \\ \hline
    6 & $0.20\pm 0.01$ & $0.204 \pm 0.009$ & $0.23 \pm 0.06$ \\ \hline
    {\cellcolor[rgb]{0.85,0.85,0.85}}7 & {\cellcolor[rgb]{0.85,0.85,0.85}}$0.199 \pm 0.006$ & {\cellcolor[rgb]{0.85,0.85,0.85}}$0.199 \pm 0.005$ & {\cellcolor[rgb]{0.85,0.85,0.85}}$0.18 \pm 0.02$ \\ \hline
    8 & $0.186 \pm 0.008$ & $0.186 \pm 0.006$ & $0.21 \pm 0.04$ \\ \hline
    {\cellcolor[rgb]{0.85,0.85,0.85}}9 & {\cellcolor[rgb]{0.85,0.85,0.85}}$0.186 \pm 0.008$ & {\cellcolor[rgb]{0.85,0.85,0.85}}$0.186 \pm 0.007$ & {\cellcolor[rgb]{0.85,0.85,0.85}}$0.237 \pm 0.04$ \\ \hline
    10& $0.207 \pm 0.003$ & $0.207 \pm 0.004 $ & $0.208 \pm 0.004$ \\ \hline \hline
    %& & & \\
    {\cellcolor[rgb]{0.85,0.85,0.85}}$E^{\ast}$ & {\cellcolor[rgb]{0.85,0.85,0.85}}$0.191\pm0.004$ & {\cellcolor[rgb]{0.85,0.85,0.85}}$0.191\pm0.004$ & {\cellcolor[rgb]{0.85,0.85,0.85}}$0.213\pm0.007$   \\ \hline
    $Err\% $ & $2$ & $2$ & $3$ \\ \hline
\end{tabular}
\end{table}

\begin{wraptable}{r}{5cm}
\caption{Compatibilità tra $E^{\ast}$}
\label{tab:compatibilità_e_star}
\centering
    \begin{tabular}{|c|c|}
        \hline
        & $\lambda$\\ \hline
        {\cellcolor[rgb]{0.85,0.85,0.85}}$\lambda_{1-2}$ & {\cellcolor[rgb]{0.85,0.85,0.85}}0.005\\ \hline
        $\lambda_{1-3}$ & 2.8\\ \hline
        {\cellcolor[rgb]{0.85,0.85,0.85}}$\lambda_{2-3}$ & {\cellcolor[rgb]{0.85,0.85,0.85}}2.8 \\ \hline
    \end{tabular}
\end{wraptable}

%Osservando l'andamento generale dei moduli di Young per ciascuno dei 3 metodi, si è notata la presenza di una misurazione poco coerente con l'andamento atteso, come si vede nel Grafico \ref{fig:andamento_e}.
Si è eseguita un a reiezione a $3\sigma$ per eliminare eventuali dati incoerenti e con i nuovi $E$ si è calcolata la media semplice $E^{\ast}$ per ciascun metodo associandovi la deviazione sulla media, riportata in Tabella \ref{tab:valori_e}. Si sono infine calcolate le compatibilità $\lambda$ reciproche fra le varie $E^{\ast}$, riportate in Tabella \ref{tab:compatibilità_e_star}.



Mediante i tre metodi riportati precedentemente si è calcolato il modulo di Young anche per gli estensimetri in tungsteno e ottone, i cui valori sono riportati in Tabella \ref{tab:tungsteno_ottone}.



%Tungsteno e Ottone
\begin{table}[h!]
    \centering
    \caption{Valori di E relativi a tungsteno e ottone}
    \label{tab:tungsteno_ottone}
    \begin{tabular}{|c|c|c|c|}
        \hline
        \multirow{2}{*}{\textbf{Estensimetro}} & \textbf{Primo metodo} & \textbf{Secondo metodo} & \textbf{Terzo metodo} \\
		& $\langle E \rangle$ $[\si{\newton\per{\micro\metre}^2}]$ &    $E_{pes}$ $[\si{\newton\per{\micro\metre}^2}]$ &  $\overline{E}$ $[\si{\newton\per{\micro\metre}^2}]$\\  \hline
        {\cellcolor[rgb]{0.85,0.85,0.85}}Tungsteno & {\cellcolor[rgb]{0.85,0.85,0.85}}$0.331 \pm 0.005$ & {\cellcolor[rgb]{0.85,0.85,0.85}}$0.33 \pm 0.02$ & {\cellcolor[rgb]{0.85,0.85,0.85}}$0.32 \pm 0.06$\\ \hline
        Ottone & $0.088 \pm 0.001$ & $0.088 \pm 0.004$ & $0.09 \pm 0.02$\\ \hline
    \end{tabular}
\end{table}

\section{Discussione dei risultati}
\subsection{Verifica dell'elasticità meccanica e definizione del miglio protocollo di misura}\label{sec:discussione_prima_parte}
\subsubsection*{Primo Metodo - Misure in allungamento e accorciamento}
Sin da una prima analisi sui dati presi dagli operatori risulta che la $\sigma_{\overline{\Delta x}}$, quando le misure risultavano tra loro identiche, è pari a zero. Per prevenire questa condizione è stato eseguito un confronto per tutte le misure tra la $\sigma_{\overline{\Delta x}}$ e la $\sigma_{Dist. uni.}$ al fine di non sottostimare l'errore. I parametri associati a tale errore risultano PTL di $\SI{10}{\micro\metre}$, in quanto ogni tacca incisa rappresenta 10 micrometri, ed un coefficiente di affidabilità pari a 10. Questa scelta è motivata dal fatto che la capacità visiva dei vari operatori risultava tale da apprezzare una variazione della posizione della lancetta di un micrometro. Si è optato per la distribuzione uniforme al posto della triangolare in quanto rappresenta più difficilmente una sottostima dell'errore.\\
Per l'incertezza su $\Delta F$ invece si è scelto come PTL il doppio dello spessore della tacca incisa in quanto si è assunto come errore massimo possibile dovuto ad un eventuale disallineamento tra indicatore e tacche incise nel dinamometro da parte dell'operatore.\\

Come già descritto nell'analisi si è operata una differenza tra le varie misurazioni di allungamento e accorciamento con la prima di queste ultime così da ricavare l'effettivo allungamento $\Delta x=| x_{i}-x_{0}|$. A ciascuna misura $\Delta x$ è stata attribuita come incertezza la propagazione degli errori ottenuta a partire dalla formula precedente.\\

Ottenuti i parametri del ${\chi}^2$ sull'allungamento e sull'accorciamento con le $\sigma_{\Delta x}$ precedentemente ottenute, si sono rieseguiti i calcoli utilizzando la sigma a posteriori. Sin da una prima analisi $\sigma_{post} \approx \SI{7.0}{\micro\meter}$ è di circa un ordine di grandezza più grande rispetto a $\sigma_{\Delta x} \approx \SI{0.6}{\micro\metre}$.

Questo controllo, pur avendo aumentato l'errore casuale sui parametri del fit, ha permesso di ottenere delle stime più attendibili e verosimili (Tabella \ref{tab:parametri_fit_1ac}). Infatti, le compatibilità fra parametri in allungamento e accorciamento con la $\sigma_{\Delta x}$ risultano incompatibili sia per il coefficiente angolare sia per l'intercetta, mentre utilizzando $\sigma_{post}$ risultano ottime.

Ipotizziamo che l'incompatibilità è relazionata ad una possibile sottostima di $\sigma_{\Delta x}$, tale da aumentare drasticamente il calcolo di $\lambda$, fenomeno che suggerisce la veridicità dell'ipotesi e il conseguente impiego di $\sigma_{post}$. In questo modo i parametri delle misure in allungamento e di quelle in accorciamento risultano pienamente compatibili.

Il confronto fra i due coefficienti angolari delle rette interpolanti dei dati in allungamento e accorciamento, mette in evidenza il comportamento elastico del filo. La compatibilità ottima infatti conferma un comportamento pressoché identico nelle differenti fasi di sollecitazione dello stesso. Va però osservato che in allungamento il coefficiente elastico risulta leggermente maggiore. Si ipotizza che questa differenza sia dovuta ad un differente comportamento elastico a seconda della sollecitazione che subisce il filo.\\

I valori delle intercette risultano tra loro compatibili ma la loro presenza è ricondotta ad un errore di natura sconosciuta, probabilmente dovuto all'andamento generale dei dati o all'errore sistematico dei quali sono affetti i dati. Ci si aspetta infatti che l'intercetta corrisponda all'origine degli assi. Tuttavia si evidenza che la compatibilità dell'intercetta con l'origine ($0\pm \sigma_{\Delta x, pos}$) è ottima ($\approx \num{0.5}$).\\

Poiché il comportamento del filo è ritenuto elastico ed essendo la compatibilità ottima, si è potuto stimare il valore di $\langle K \rangle \pm \sigma_{\langle K \rangle}$ e considerarlo come coefficiente di elasticità rappresentativo per il filo.\\

\subsubsection*{Secondo Metodo - Misure ripetute}
\begin{figure}[h!]
    \centering
    \subfloat[Andamento misure a $\SI{400}{gp}$]{
        \includegraphics[width=0.5\textwidth]{tutto_400.png}
        \label{fig:tutto_400}
        }
    \subfloat[Andamento misure a $\SI{1000}{gp}$]{
        \includegraphics[width=0.5\textwidth]{tutto_1000.png}
        \label{fig:tutto_1000}
        }
\end{figure}

La possibilità di compiere misure ripetute mette in evidenza le differenze che intercorrono tra le misurazioni raccolte durante la fase di allungamento e di accorciamento. I Grafici \ref{fig:tutto_400} e \ref{fig:tutto_1000} infatti mostrano che le misure prese in accorciamento risultano essere maggiori di quelle prese in allungamento, fenomeno che si ripresenta per $F_{letta}=\SI{400}{gp}$ che per $F_{letta}=\SI{1000}{gp}$ ma anche la presenza di evidenti dispersioni. Le differenze fra gli errori sono dovute a differenti letture da parte degli operatori. La prima barra di errore del Grafico \ref{fig:tutto_400} risulta essere maggiore delle altre, quindi si suppone una errata lettura da parte di un operatore.

La prima evidenza sopra riportata riscontrabile nella rappresentazione grafica delle misurazioni mostra che i dati raccolti in allungamento, sia a $\SI{400}{gp}$ che $\SI{1000}{gp}$, risultano essere minori di quelli in accorciamento. Spiegazione di tale fenomeno è attribuibile alla presenza di un errore sistematico dovuto allo stress meccanico subito dal filo durante entrambe le fasi di sollecitazione. E' importante osservare che l'errore sistematico influenza sia le misure prese in accorciamento che in allungamento.\\
Un'altra evidenza non facilmente riscontrabile è il sistematico aumento del valore delle misurazioni con il succedersi delle misure stesse. Nei grafici \ref{fig:tutto_400} e \ref{fig:tutto_1000} è stata rappresentata una retta che interpolasse i dati presi sia in allungamento che in accorciamento, rappresentate nel grafico. Il coefficiente angolare positivo mostra la tendenza delle misurazioni ad aumentare progressivamente. Si ipotizza pertanto che un ripetuto allungamento e accorciamento del filo vadano ad alterarne il comportamento elastico. Si osserva tuttavia un'eccezione alla retta nel Grafico \ref{fig:tutto_400} relativa alle misure in allungamento. La pendenza negativa è probabilmente attribuibile soltanto alle ultime tre misure, da considerare forse affette da un ulteriore errore sistematico subentrato solamente in questi ultimi tre casi.

In generale tuttavia, l'ipotesi di questo sistematico aumento delle misurazioni trova fondamento nella specularità degli andamenti delle misure appartenenti ai processi di distensione e contrazione. La spiegazione di questo fenomeno sembra essere attribuibile alla presenza di una memoria meccanica del materiale.\\

A seguito di queste considerazioni si è ipotizzato che le varie misurazioni non fossero statisticamente indipendenti come osservato precedentemente, dunque invece di utilizzare la media semplice si è utilizzata la media ponderata e l'errore sulla media ponderata per calcolare un valore rappresentativo di ciascun campione.\\

$\langle \Delta x \rangle \pm \sigma_{\langle \Delta x \rangle}$ sono stati impiegati per il calcolo dei vari K. Si noti che si sono tenute separate le misure eseguite in allungamento e accorciamento il più possibile, fino a quando non si è eseguita la media ponderata  fra ${\langle K \rangle}_{acc}$ e ${\langle K \rangle}_{all}$. Questa accortezza è dovuta ad una scelta arbitraria compiuta a priori nel processo di analisi dati, in modo da risultare coerente con l'obiettivo di mostrare una possibile componente di errore sistematico.\\

Una possibile stima dell'errore sistematico è stata computata sulla differenza di allungamento e accorciamento, come mostrato nel paragrafo \ref{par:errore_desalva}. Si osservi che l'errore stimato risulti essere molto inferiore di dimensione rispetto alla stima di $\langle K \rangle$ effettuata precedentemente. 

\subsubsection*{Definizione miglior protocollo misura}

\begin{table}[h!]
    \centering
    \begin{tabular}{|c|c|c|c|c|c|c|}
        \hline
        \multicolumn{2}{|c|}{} &$K_{all}\pm\sigma_{K_{all}}$& $K_{acc}\pm\sigma_{K_{acc}}$& $K \pm \sigma_K$ & \multirow{2}{*}{$\lambda$} & \multirow{2}{*}{Err\%}\\ 
        \multicolumn{2}{|c|}{} & $[\si{\micronnewton}]$ & $[\si{\micronnewton}]$ &$[\si{\micronnewton}]$ & &\\ \hline
        \multicolumn{2}{|l|}{1 acquisizione} &{\cellcolor[rgb]{0.85,0.85,0.85}}$55.3\pm0.2$ &{\cellcolor[rgb]{0.85,0.85,0.85}}$55.17\pm0.09$ &{\cellcolor[rgb]{0.85,0.85,0.85}} $55.20\pm0.02$ &{\cellcolor[rgb]{0.85,0.85,0.85}} 0.7 &{\cellcolor[rgb]{0.85,0.85,0.85}} 0.04\\ \hline
        \multirow{2}{*}{2 acquisizione} & 1 metodo & $56.32\pm0.08$ & $55.68\pm0.08$ & $55.99\pm0.05$ & 5.5 & 0.1\\ \cline{2-7}
         & 2 metodo &{\cellcolor[rgb]{0.85,0.85,0.85}} $55.42\pm0.06$ &{\cellcolor[rgb]{0.85,0.85,0.85}} $55.68\pm0.07$ & {\cellcolor[rgb]{0.85,0.85,0.85}} $55.53\pm0.05$ &{\cellcolor[rgb]{0.85,0.85,0.85}} 2.7 &{\cellcolor[rgb]{0.85,0.85,0.85}} 0.08\\ \hline
    \end{tabular}
    \caption{Protocolli di misura}
    \label{tab:protocolli}
\end{table}

Come si può riscontrare nella tabella \ref{tab:protocolli}, è presente un'evidente differenza sulle $\sigma_K$ calcolate per i i due metodi differenti. Per i procedimenti utilizzati nella seconda acquisizione, ovvero la presa dati di misure ripetute, si riscontra una differenza percentuale media di circa 0.05\%  rispetto al primo metodo, ovvero alle misure in accorciamento e allungamento. 

Ipotizziamo inoltre che la scelta del miglior protocollo di misura ricada sul primo metodo e non sul secondo in quanto il continuo stress al quale è sottoposto il filo nella seconda acquisizione, ovvero nelle misure ripetute, influenzi notevolmente l'andamento dei dati, cosa che nel primo metodo non è riscontrata con i dati a disposizione. La dimostrazione di tale affermazione è da ricercarsi nel confronto effettuato tra le compatibilità calcolate tra  $K_{All}$ e $K_{Acc}$ della prima acquisizione e la compatibilità calcolata tra la media ponderata di $K_{All, 400}$ e $K_{All, 1000}$ con $K_{Acc, 400}$ e $K_{Acc, 1000}$. Il valore di $\lambda$ calcolato per la seconda acquisizione per entrambi i metodi risulta essere incompatibile e dunque ciò dimostra una presenza notevole dell'errore sistematico nel secondo metodo piuttosto che nel primo.

\subsection{Analisi sistematica di tutti gli estensimetri}
\subsubsection*{Calcolo di K}
Come già descritto nell'analisi si è operata una differenza tra le varie misurazioni di allungamento e accorciamento $\Delta x= |x_i - x_0|$ associandovi il relativo errore dalla propagazione.

L'errore sulle singole $x_i$ invece è stato derivato dalla distribuzione uniforme utilizzando come PTL $\SI{5}{\micro\meter}$ e come coefficiente di affidabilità 1. Confrontando la retta interpolante generata dal $\chi^2$ si è osservato un andamento dei dati piuttosto disperso e ciò ha suggerito di valutare l'errore $\sigma_{\Delta x, post}$. Il calcolo di quest'ultimo ha infatti fatto emergere una sottostima dell'errore dei singoli $\Delta x$, poi sostituiti con la stima a posteriori. $\sigma_{\Delta x, post}$ risultava $\approx 20$ volte superiore alla stima dell'errore precedentemente usata. (Tabella \ref{tab:confronto_sigma_posteriori})

\begin{wraptable}{l}{6cm}
\centering
\caption{Confronto tra $\sigma_{post}$}
\label{tab:confronto_sigma_posteriori}
    \begin{tabular}{|c|c|c|}
    \hline
    \multirow{2}{*}{Est} & $\sigma_{post_{all}}$ & $\sigma_{post_{acc}}$ \\
        & $[\si{\micronnewton}]$ & $[\si{\micronnewton}]$  \\\hline
    {\cellcolor[rgb]{0.85,0.85,0.85}}1   & {\cellcolor[rgb]{0.85,0.85,0.85}}87.2       & {\cellcolor[rgb]{0.85,0.85,0.85}}84.5        \\\hline
    2   & 105.6       & 105.3        \\\hline
    {\cellcolor[rgb]{0.85,0.85,0.85}}3   & {\cellcolor[rgb]{0.85,0.85,0.85}}50.6       & {\cellcolor[rgb]{0.85,0.85,0.85}}60.3        \\\hline
    4   & 90.2       & 99.2        \\\hline
    {\cellcolor[rgb]{0.85,0.85,0.85}}5   & {\cellcolor[rgb]{0.85,0.85,0.85}}93.4       & {\cellcolor[rgb]{0.85,0.85,0.85}}100.5        \\\hline
    6   & 71.6       & 72.3        \\\hline
    {\cellcolor[rgb]{0.85,0.85,0.85}}7   & {\cellcolor[rgb]{0.85,0.85,0.85}}73.2       & {\cellcolor[rgb]{0.85,0.85,0.85}}69.6         \\\hline
    8   & 80.2       & 68.4        \\\hline
    {\cellcolor[rgb]{0.85,0.85,0.85}}9   & {\cellcolor[rgb]{0.85,0.85,0.85}}101.4       & {\cellcolor[rgb]{0.85,0.85,0.85}}105.6        \\\hline
    10  & 7.1       & 3.6        \\\hline\hline
    {\cellcolor[rgb]{0.85,0.85,0.85}}11   & {\cellcolor[rgb]{0.85,0.85,0.85}}134.7      & {\cellcolor[rgb]{0.85,0.85,0.85}}128.6       \\\hline
    12  & 115.9      & 117.1       \\\hline

    \end{tabular}
\end{wraptable}

I grafici \ref{fig:1_estensimetro} -- \ref{fig:12_estensimetro} evidenziano una regione di confidenza dove si è tenuto conto dell'errore a posteriori, che ricopre tutti i dati e gli errori precedentemente valutati ed evidenza la sostanziale differenza tra le due differenti stime.

Si ipotizza che i dati siano in realtà affetti oltre che dall'errore casuale derivante dalla lettura della lancetta sul minimetro da parte dell'operatore, anche ad una componente casuale maggiore, relativa forse a giochi meccanici del minimetro preso in considerazione. Si ritiene che proprio i giochi meccanici siano responsabili delle fluttuazioni delle misure intorno alla retta interpolante.\\
Si è valutata poi la compatibilità che intercorre tra i valori delle intercette delle rette calcolate per ciascun grafico con l'origine degli assi in quanto ci si aspetta che il valore dell'intercetta stessa sia pari a zero. Come si evidenza in Tabella \ref{tab:k_est} la compatibilità sono sempre ottime. Si specifica che è stato usato $\sigma_{\Delta x, post}$ come errore nell'origine degli assi. Il fatto che ci siano compatibilità ottime dimostra la bontà dei dati.\\
Un commento a parte meritano i Grafici \ref{fig:11_estensimetro} e \ref{fig:12_estensimetro} riferiti agli estensimetri in tungsteno ed ottone. La $\sigma_{post}$ risulta essere sia in allungamento che in accorciamento molto grande. Sin da una analisi grafica infatti emergono dei dati molto dispersi. In particolare si sospetta un errore nella terza misura, forse attribuibile ad un errore di lettura dell'operatore. Si osservi infine come le misurazioni corrispondenti a $\Delta F$ maggiore non rispettino l'andamento della retta, ma aumentino di poco rispetto a quelle precedenti. Si ipotizza che da quel punto in poi vi sia un malfunzionamento del minimetro tale da comportare un'errata lettura. Ipotesi alternativa, verificabile solo con la presenza di più dati, è un cambiamento del comportamento elastico del metallo. 


\subsection{Verifica della legge}
I metodi definiti per dimostrare la validità della legge utilizzata nel paragrafo \S \textit{ Stima di E} ha evidenziato una concordanza tra la teoria e i dati acquisiti, tranne in alcuni casi dove è stato necessario approfondire l'analisi dati.\\
Per il primo metodo infatti risulta che il coefficiente angolare della retta di interpolazione dati risulta essere positivo, e ciò dimostra la diretta dipendenza dei parametri K e $x_0$ presenti nella formula riportata nell'analisi. Si suppone però che la valutazione non sia accurata, data la dimensione ridotta del campione.
Bisogna sottolineare che il punto avente $\sigma_K$ minore rappresenta l'estensimetro 10, i cui dati sono stati presi manualmente come descritto nella prima parte della relazione, avente lunghezza a riposo di $\SI{700}{\milli\meter}$.\\


Per il secondo metodo, analogamente al primo, si è dimostrata la diretta proporzionalità tra i parametri K e 1/S  valutando la positività del coefficiente angolare della seconda retta interpolante, rappresentata in rosso. Il motivo della reiezione del dato, come descritto ampiamente nell'analisi, è da ricercarsi nelle caratteristiche intrinseche del filo, siano esse i parametri che lo descrivono o la presa dati.
Dalla tabella dei dati grezzi si identifica infatti un anomalo comportamento dell'estensimetro numero 5 le cui caratteristiche sono descritte in Tabella \ref{tab:caratteristiche_estensimetri}. Con l'aumentare della forza applicata si osserva che l'allungamento del filo risulta essere visivamente minore rispetto all'andamento delle misurazioni in estensimetri aventi $x_{0}$ e sezione simili. 
Si ipotizza pertanto che ci sia un errore o nei parametri o nelle misure di allungamento ed accorciamento o addirittura in entrambi. L'errore nei parametri potrebbe essere riconducibile ad un errore di trascrizione nelle tabelle, mentre l'errore nelle misurazioni potrebbe essere causato da un malfunzionamento del minimetro.\\


Il terzo metodo consiste nel verificare se il calcolo del rapporto R rimane costante per estensimetri aventi la stessa sezione, al variare della lunghezza a riposo di ciascun filo. La legge risulta essere pienamente verificata in quanto il valore del coefficiente angolare è $\approx 0$, ma anche in questo caso la dimensione del campione, ovvero di 4 dati rende impossibile poter verificare con certezza tale affermazione.\\


Per il calcolo del prodotto P invece è stato utilizzato un campione di dati più grande e anche in questo caso si è dimostrata la validità della legge analizzando il valore del coefficiente angolare della retta interpolante i valori ottenuti. Nel campione di dati però compare un dato anomalo che, in modo analogo a quanto precedentemente descritto, è stato reiettato. Il dato infatti corrisponde all'estensimetro 5, lo stesso estensimetro scartato nella verifica della diretta proporzionalità fra K e $x_{0}$.\\

\subsubsection*{Stima di E}
Le stime del modulo di Young sono state effettuate, come descritto nell'analisi, con l'utilizzo di tre metodi differenti. Ciò permette di valutare eventuali contributi di errore derivanti dall'utilizzo di un unica metodologia per il calcolo di E ed inoltre permette di analizzare e comparare i dati ottenuti da i diversi metodi tra di loro. Il computo di E con diversi metodi ha fatto emergere una comune incongruenza di un dato, infatti come accennato nel Grafico \ref{fig:andamento_e} si è resa necessaria una reiezione a $3\sigma$ in quanto un dato non rispettava l'andamento generale. Il dato reiettato è riferito allo stesso estensimetro eliminato dai fit nei grafici precedenti, il numero 5.\\
Si riporta il Grafico \ref{fig:gaussiana} rappresentante le distribuzioni dei moduli di $E^{\ast}$ derivati dal primo metodo. Si può osservare la curva gaussiana caratteristica ed in rosso il dato reiettato.

\begin{figure}[h!]
    \centering
        \caption{Andamento E con i 3 metodi}
        \label{fig:andamento_e}
        \includegraphics[width=0.8\textwidth]{andamento_E.png}
\end{figure}


\begin{figure}[h!]
    \centering
    \caption{Distribuzione $E^{\ast}$ con il primo metodo per gli estensimetri in acciaio}
    \label{fig:gaussiana}
    \includegraphics[width=12.5cm]{gaussiana.png}
\end{figure}

La differenza sostanziale fra i due metodi viene evidenziata dall'errore del quale sono affette le misure del terzo metodo. Gli errori calcolati con quest'ultimo metodo sono molto maggiori rispetto ai primi due metodi. Ciò accade in quanto $\sigma_k$ è maggiore che negli altri metodi. Questo si verifica perché essa deriva dalla $\sigma_{\overline{K}}$ sui campioni di K non consecutivi che a loro volta sono molto dispersi perché derivanti da misure molto disperse attorno alla retta interpolante. Questa ultima affermazione è un'ulteriore prova della dispersione dei dati per i primi 9 estensimetri. Infatti il decimo estensimetro presenta errori comparabili per tutti e tre i metodi di stima di $E^{\ast}$ in quanto la dispersione dei $\Delta x$ di allungamento e accorciamento è significativamente inferiore a quella degli altri estensimetri analizzati.\\
A seguito di questa considerazione il metodo peggiore per stimare $E^{\ast}$  risulta essere il terzo in quanto più fortemente influenzato dalle misurazioni dei $\Delta x$. Gli altri due metodi infatti si avvalgono di una stima derivata dai contributi di tutte le stime, ovvero con il chi quadro. A dimostrazione di ciò si confrontino le compatibilità fra i 3 metodi. $\lambda_{1,3}$ e $\lambda_{2,3}$ che risultano rispettivamente pessima ed incompatibile. $\lambda_{1,2}$ invece è discreta. A parità di efficacia si sceglie il metodo che genera errore \% minore sul modulo di Young. Primo e secondo metodo hanno errori minori e tra loro differenti a meno dello 0.02\%. Pertanto si possono considerare come metodi per la stima del modulo di Young del tutto equivalenti.\\
Non è possibile compiere confronti analoghi per gli estensimetri in tungsteno ed ottone. Si può osservare tuttavia che il terzo metodo come precedentemente accennato restituisce un'incertezza maggiore poiché i dati mostrati dai grafici sono più dispersi. Tra i tre metodi si considera pertanto quello con errore casuale minore.

\section{Margini di miglioramento}
Durante l'analisi e la discussione dell'esperienza si sono individuati alcuni margini di miglioramento al fine di ridurre gli errori nelle misurazioni e ottenere risultati più accurati e precisi.
\paragraph{Strumentazione}
Lo strumento del minimetro analogico potrebbe essere una causa rilevante di errore in quanto su quelle misurazioni si è basata tutta l'esperienza. L'utilizzo di uno strumento digitale, con un errore molto inferiore su ogni singola misurazione avrebbe di certo minimizzato l'errore casuale di ogni singola misurazione.
\paragraph{Grandezza dei campioni analizzati}
L'analisi dei dati, nella maggioro parte di casi, non ha dimostrato appieno ciò che la teoria suggeriva a causa del numero ristretto di campioni analizzati. Si suppone che un campione più grande di estensimetri avrebbe determinato maggiore precisione nella stima finale del modulo di Young, e anche che campioni più grandi di dati nella seconda acquisizione avrebbe determinato una statistica più accurata.

\newpage
\section{Conclusione}
L'obiettivo dell'esperienza è stato pienamente raggiunto, si è verificata l'elasticità dei fili d'acciaio, di tungsteno ed ottone. Si è definito il miglior protocollo di misura, ovvero quello riportato nella sezione \ref{sec:miglior_metodo_misura}. Le stime relative ai coefficienti elastici dei vari estensimetri vengono riportati in Tabella \ref{tab:k_est} e la stima dei moduli di Young per ciascun materiale sono riassunti in Tabella \ref{tab:young_finale}.

\begin{table}[h!]
    \centering
    \begin{tabular}{|c|c|c|}
     \hline
        \multirow{2}{*}{Materiale}& E^{\ast}& \multirow{2}{*}{$Err\%$}\\
        &$[\si{\newton\per\micro\meter\squared}]$&\\ \hline
       {\cellcolor[rgb]{0.85,0.85,0.85}}Acciaio& {\cellcolor[rgb]{0.85,0.85,0.85}}$0.191\pm0.004$ &	{\cellcolor[rgb]{0.85,0.85,0.85}}$2$\\
       Tungsteno& $0.331\pm0.005$ &	$1.5$\\
       {\cellcolor[rgb]{0.85,0.85,0.85}}Ottone& {\cellcolor[rgb]{0.85,0.85,0.85}}$0.088\pm0.001$ &	{\cellcolor[rgb]{0.85,0.85,0.85}}$1.1$\\
    \hline
    \end{tabular}
    \caption{Moduli di Young stimati per i diversi metalli}
    \label{tab:young_finale}
\end{table}




\section{Appendice}

\subsection{Formulario}
\textbf{Media, deviazione standard, deviazione standard della media}
\begin{align*}
   % \begin{aligned}
        \overline{x}&=\sum\limits_{i=1}^{N} \frac{x_{i}}{N}&
        \sigma&=\sqrt{\frac{\sum\limits_{i=1}^{N} (x_{i}-\overline{x})}{N-1}}&
        \sigma_{\overline{x}}&=\frac{\sigma}{\sqrt{N}}
   % \end{aligned}
\end{align*}\\

\textbf{Media Ponderata}
\begin{equation*}
\label{eq:media_pond}
    x_i=\frac{\sum_{i=1}^{N}\frac{x_i}{\sigma_{x_i}}}{\sum_{i=1}^{N}\frac{1}{\sigma_{x_i}}}
\end{equation*}

\textbf{Errore Media Ponderata}
\begin{equation*}
\label{eq:errore_media_pond}
     \sigma_{x_i}=\sqrt{\frac{1}{\sum_{i=1}^{N}\frac{1}{\sigma_{i}^{2}}}}
\end{equation*}

\textbf{Formule per il ${\chi}^2$}
\begin{equation*}
        \begin{cases}
    a=&\frac{1}{\Delta}[(\sum\limits_{i=1}^{N}{x_{i}^{2}})\cdot(\sum\limits_{i=1}^{N}{y_{i}})-(\sum\limits_{i=1}^{N}{x_{i}})\cdot(\sum\limits_{i=1}^{N}{x_{i}y_{i}})] \\ 
    b=&\frac{1}{\Delta }\cdot \left [N\cdot \left ( \sum\limits_{i=1}^{N}x_i y_i \right )-\left ( \sum\limits_{i=1}^{N}x_i \right )\cdot \left ( \sum\limits_{i=1}^{N}y_i \right )  \right ]\\
    \Delta=& N\cdot \sum\limits_{i=1}^{N} x_i^{2} - \left ( \sum\limits_{i=1}^{N}x_i \right )^{2}\\
    \end{cases}
\end{equation*}
\begin{equation*}
    \begin{cases}
    \sigma_{a}=&\sigma_{y}\cdot\sqrt{\frac{\sum_{i=1}^{N}{x_{i}^{2}}}{\Delta}} \\
    \sigma_{b}=&\sigma_y\cdot \sqrt{\frac{N}{\Delta }}\\
    \end{cases}
    \label{equation:err_chi_quadro}
\end{equation*}
\\
\textbf{Formula di propagazione degli errori casuali}\\

Sia z=($x_1$,...;$x_N$) funzione di N variabili casuali $x_1$,...,$x_N$ e sia ${x_i^\ast}$=($x_1^\ast$,...,$x_N^{\ast}$) l'insieme di tutti i valori veri associati a tali variabili, si ha 

\begin{equation*}
    \sigma_z^{2}\approx  \sum_{i=j=1}^{N}\left ( \frac{\partial z}{\partial x_i}\Big|_{x_i^{\ast}} \right )^{2}\cdot\sigma_{x_i}^{2} +\sum_{i=1,j=1,i\neq j}^{N}\left (\frac{\partial z }{\partial x_i}\Big|_{x_i^{\ast}} \right ) \cdot \left ( \frac{\partial z}{\partial x_j} \Big|_{x_j^{\ast}} \right )\cdot cov(x_i,x_j)\label{eq:prop_errori}
\end{equation*}
E' stato utilizzato il simbolo $\approx$ in quanto si è scelto di troncare al primo termine lo sviluppo in serie di Taylor.\\


\textbf{Formula calcolo compatibilità}\\
\begin{equation*}
    \lambda=\frac{\left|a-b\right|}{\sqrt{\sigma^{2}_{a}+\sigma^{2}_{b}}}
\end{equation*}\\

%plot "primo_metodo.txt" u 1:2:3 w yerrorbars, "secondo_metodo.txt" u 1:2:3 w yerrorbars lc 7, "terzo_metodo.txt" u 1:2:3 w yerrorbars lc 14

\subsection{Codice sorgente}
\begin{lstlisting}[language=C++, label=lst:statistica.h, caption=statistica.h]
#include <iostream>
#include <fstream>
#include <string>
#include <vector>
#include <cmath>
#include <cctype>
using namespace std;

//media
double media(vector<double> dati, int inizio = 0, int fine = 0)
{
    if (fine != 0)
    {
        double media_, counter_parz = 0, sum_parz = 0;
        for (int i = inizio; i < fine; i++)
        {
            sum_parz = sum_parz + dati[i];
            counter_parz++;
        }
        media_ = sum_parz / counter_parz;
        return media_;
    }
    else if (fine == 0)
    {
        double media_, sum = 0;
        for (auto &c : dati)
        {
            sum = sum + c;
        }
        media_ = sum / dati.size();
        return media_;
    }
}
//Deviazione standard per popolazione (N-1 a denominatore)
double dstd(vector<double> dati, int inizio = 0, int fine = 0, string log = "")
{
    double dstd_camp;
    double media_camp;
    double somma_camp;
    int counter = 0;
    media_camp = media(dati, inizio, fine);
    for (auto c : dati)
    {
        somma_camp = somma_camp + (media_camp - c) * (media_camp - c);
    }
    counter = dati.size();
    dstd_camp = sqrt(somma_camp / (counter - 1));

    return dstd_camp;
}

//Deviazione standard della media
double dstd_media(vector<double> dati)
{
    double dstd_media_;

    dstd_media_ = dstd(dati) / sqrt(dati.size());

    return dstd_media_;
}

//Errore distribuzione triangolare
double sigma_dist_tri(double ptl, double coeff_aff)
{
    return abs(ptl / coeff_aff) / sqrt(24); 
}

//Errore distribuzione uniforme
double sigma_dist_uni(double ptl, double coeff_aff)
{
    return abs(ptl / coeff_aff) / sqrt(12); 
}

//Delta (chi-quadro) [errori tutti uguali o del tutto assenti]
double delta(vector<double> dati_x)
{
    double delta_ = 0;
    double size = 0;
    double sum_1 = 0, sum_2 = 0;

    size = dati_x.size();
    for (auto &d : dati_x)
    {
        sum_1 = sum_1 + pow(d, 2);
        sum_2 = sum_2 + d;
    }
    delta_ = size * sum_1 - pow(sum_2, 2);

    return delta_;
}

//Errore intercetta se tutti i sigma sono uguali
double sigma_a_y_uguali(vector<double> dati_x, double err_y)
{
    double sum1 = 0, sum2 = 0, sum = 0;
    for (int i = 0; i < dati_x.size(); i++)
    {

        sum1 = sum1 + pow(dati_x[i], 2);
    }

    return err_y * sqrt(sum1 / delta(dati_x));
}

//Errore coeff. ang. se tutti i sigma y sono uguali
double sigma_b_y_uguali(vector<double> dati_x, double err_y)
{
    double delta_ = delta(dati_x);
    double size = dati_x.size();
    return err_y * sqrt(size / delta_);
}

//Media ponderata con errori
double media_ponderata(vector<double> valori, vector<double> errori, int inizio = 0, int fine = 0)
{
    double num = 0, den = 0;
    if (fine == 0)
    {
        fine = valori.size();
    }
    if (valori.size() != errori.size())
    {
        cout << "dimensione di vettore valori non è uguale a quella di vettore errori" << endl;
        return 1;
    }
    for (int i = inizio; i < fine; i++)
    {
        num += valori[i] * pow((1 / errori[i]), 2);
        den += pow((1 / errori[i]), 2);
    }
    return num / den;
}

//Coefficiente a di y=a+bx (intercetta)
double a_intercetta(vector<double> dati_x, vector<double> dati_y)
{
    double coeff_a = 0;
    double delta_ = 0;
    double sum_1 = 0, sum_2 = 0, sum_3 = 0, sum_4 = 0;
    if (dati_y.size() != dati_x.size())
    {
        cout << "Dimensione di vettore per dati ascisse è diversa da dimensione vettore dati ordinate" << endl;
    }

    for (int i = 0; i < dati_x.size(); i++)
    {
        sum_1 = sum_1 + pow(dati_x.at(i), 2);
        sum_2 = sum_2 + dati_y.at(i);
        sum_3 = sum_3 + dati_x.at(i);
        sum_4 = sum_4 + (dati_x.at(i) * dati_y.at(i));
    }
    delta_ = delta(dati_x);
    coeff_a = (1 / delta_) * (sum_1 * sum_2 - sum_3 * sum_4);

    return coeff_a;
}

//Coefficiente b di y=a+bx (coeff. angolare)
double b_angolare(vector<double> dati_x, vector<double> dati_y)
{
    double coeff_b = 0;
    double delta_ = 0;
    int size = 0;
    double sum_1 = 0, sum_2 = 0, sum_3 = 0;
    if (dati_y.size() != dati_x.size())
    {
        cout << "Dimensione di vettore per dati ascisse è diversa da dimensione vettore dati ordinate";
    }

    for (int i = 0; i < dati_x.size(); i++)
    {
        sum_1 = sum_1 + dati_x.at(i) * dati_y.at(i);
        sum_2 = sum_2 + dati_x.at(i);
        sum_3 = sum_3 + dati_y.at(i);
    }
    delta_ = delta(dati_x);
    size = dati_x.size();
    coeff_b = (1 / delta_) * (size * sum_1 - sum_2 * sum_3);

    return coeff_b;
}

//Errore medio (media ponderata)
double errore_media_ponderata(vector<double> errori)
{
    double sum = 0;
    for (auto &d : errori)
    {
        sum += pow((1 / d), 2);
    }
    return (1 / sqrt(sum));
}
//Sigma y a posteriori
double sigma_y_posteriori(vector<double> dati_x, vector<double> dati_y)
{
    double sigma_y;
    double a, b;
    int size = 0;
    double numeratore = 0;
    if (dati_x.size() != dati_y.size())
    {
        cout << "Vettori forniti non della stessa dimensione" << endl;
    }

    a = a_intercetta(dati_x, dati_y);
    b = b_angolare(dati_x, dati_y);
    size = dati_x.size();
    for (int i = 0; i < dati_x.size(); i++)
    {
        numeratore = numeratore + pow((dati_y.at(i) - a - b * dati_x.at(i)), 2);
    }
    sigma_y = sqrt(numeratore / (size - 2));
    return sigma_y;
}

//Compatibilità avendo due valori medi e i sigma relativi
double comp_3(double a, double b, double sigma_a, double sigma_b)
{
    return abs(a - b) / sqrt(pow(sigma_a, 2) + pow(sigma_b, 2));
}


void chi_quadro(vector<double> dati_x, vector<double> dati_y, vector<double> err_y)
{
    double a_ = a_intercetta(dati_x, dati_y);
    double b_ = b_angolare(dati_x, dati_y);
    double sum = 0;

    for (int i; i < dati_x.size(); i++)
    {
        cout << pow((((a_intercetta(dati_x, dati_y) + b_angolare(dati_x, dati_y) * dati_x[i]) - dati_y[i])) / err_y[i], 2);
    }
    
    return;
}
\end{lstlisting}

\begin{lstlisting}[language=C++, label=lst:analisi_1ac, caption=analisi\_1ac]
#include <iostream>
#include <cmath>
#include <fstream>
#include <string>
#include <vector>
#include "statistica1.h"
using namespace std;

double kgpeso_to_newton(double kg_peso)
{
    return 9.806 * kg_peso;
}

int main()
{
    string dir_ac, sub_dir, alto_basso;
    double t;
    string numero;

    vector<double> delta_forza_newton; 
    for (double i = 200; i < 1300; i = i + 100)
    {
        delta_forza_newton.push_back(kgpeso_to_newton((i - 200) * 4 / 1000)); 
    }

    dir_ac = "1ac"; 

    vector<double> media_1_acq, dstd_1_acq;
    cout << "Vuoi legge i file di Allungamento o Accorciamento? (all / acc): ";
    cin >> sub_dir;
    ofstream excelout("../operatorimisure.txt", ios::app);
    for (int i = 200; i < 1300; i = i + 100) 
    {
        vector<double> dati_file; 
        numero = to_string(i);    
        ifstream fin("../" + dir_ac + "/" + sub_dir + "/" + numero + ".txt");
        if (!fin) 
        {
            cout << "Errore di lettura" << endl;
            return 1;
        }
        while (fin >> t)
        {
            dati_file.push_back(t * 10.0); 
        }
        media_1_acq.push_back(media(dati_file));
        dstd_1_acq.push_back(dstd_media(dati_file));
        excelout << dati_file[0]<<"\t"<<dati_file[1]<<"\t"<<dati_file[2]<<"\t"<<media(dati_file) << "\t" << dstd(dati_file) << "\t" << dstd_media(dati_file) << endl;
    }

    for (int i = 0; i < dstd_1_acq.size(); i++)
    {
        if (dstd_1_acq[i] < sigma_dist_uni(10, 10))
        {
            dstd_1_acq[i] = sigma_dist_uni(10, 10);
        }
    }

    cout << "POST\tERR_NS\tDIST_UNI" << endl;
    for (int i = 0; i < dstd_1_acq.size(); i++)
    {
        cout << sigma_y_posteriori(delta_forza_newton, media_1_acq) << "\t" << dstd_1_acq[i] << endl;
    }
    vector<double> errori_y(dstd_1_acq.size(), sigma_y_posteriori(delta_forza_newton, media_1_acq));

    
    ofstream fout("../Grafici/1ac_all_acc/grafico_" + sub_dir + ".txt");
    ofstream dout("comp_1ac_correz_dist_uni.txt", ios::app);
    ofstream sout("comp_1ac_sigma_post.txt", ios::app);

    if (!fout)
    {
        cout << "Errore scrittura";
        return 1;
    }
    fout << "#DeltaForza[N]\tDelta_x[1e-6m]\tDstd(y_post)[1e-6m]" << endl; 
    for (int i = 0; i < media_1_acq.size(); i++)                           
    {
        fout << delta_forza_newton[i] << "\t" << media_1_acq[i] << "\t" << sigma_y_posteriori(delta_forza_newton, media_1_acq) << endl;
    }
    
    cout << "Err.Piccoli (sottostima) Coeff ang:\t" << b_angolare(delta_forza_newton, media_1_acq, dstd_1_acq) << " +/- " << sigma_b(delta_forza_newton, media_1_acq, dstd_1_acq) << endl;
    cout << "Sig. Post. (funzioni senza err) Coeff ang:\t" << b_angolare(delta_forza_newton, media_1_acq) << " +/- " << sigma_b(delta_forza_newton, media_1_acq) << endl;

    cout << "Err.Piccoli (sottosti Intercetta:\t" << a_intercetta(delta_forza_newton, media_1_acq, dstd_1_acq) << " +/- " << sigma_a(delta_forza_newton, media_1_acq, dstd_1_acq) << endl;
    cout << "Sig. Post. (funzioni  Intercetta:\t" << a_intercetta(delta_forza_newton, media_1_acq) << " +/- " << sigma_a(delta_forza_newton, media_1_acq) << endl;
    cout<<"COmpatibilità intercetta con origine"<<comp_3(a_intercetta(delta_forza_newton, media_1_acq), 0, sigma_a(delta_forza_newton, media_1_acq), sigma_y_posteriori(delta_forza_newton, media_1_acq))<<endl;

    
    dout << b_angolare(delta_forza_newton, media_1_acq, dstd_1_acq) << "\t" << sigma_b(delta_forza_newton, media_1_acq, dstd_1_acq) << "\t" << a_intercetta(delta_forza_newton, media_1_acq, dstd_1_acq) << "\t" << sigma_a(delta_forza_newton, media_1_acq, dstd_1_acq) << endl;
    sout << b_angolare(delta_forza_newton, media_1_acq) << "\t" << sigma_b(delta_forza_newton, media_1_acq) << "\t" << a_intercetta(delta_forza_newton, media_1_acq) << "\t" << sigma_a(delta_forza_newton, media_1_acq) << endl;
    return 0;
}
\end{lstlisting}

\begin{lstlisting}[language=C++, label=lst:analisi_2ac, caption=analisi\_2ac]
#include <iostream>
#include <cmath>
#include <fstream>
#include <string>
#include <vector>
#include "statistica.h"
using namespace std;

double kgpeso_to_newton(double kg_peso);

struct container
{
  string nomefile = "";
  vector<double> temp;
  vector<double> media_2_acq;
  vector<double> dstd_2_acq;
  double media;
  double dstd_media;
  double kappa;
  double sigma_kappa;
  double gp = 0;
};
vector<container> campioni(4);

int main()
{
  double t;
  string numero;
  vector<double> k_alto;
  vector<double> k_basso;
  vector<double> sigma_k_alto;
  vector<double> sigma_k_basso;

  campioni[0].nomefile = "../2ac/400/400_basso.txt";
  campioni[1].nomefile = "../2ac/400/400_alto.txt";
  campioni[2].nomefile = "../2ac/1000/1000_basso.txt";
  campioni[3].nomefile = "../2ac/1000/1000_alto.txt";
  campioni[0].gp = 400.0;
  campioni[1].gp = 400.0;
  campioni[2].gp = 1000.0;
  campioni[3].gp = 1000.0;
  double sigma_delta_forza = sigma_dist_uni(((300.0 * 100.0) / (11.0 * 1000.0)) * (4.0 * 9.806 / 1000.0), 1.0);

  for (auto &d : campioni)
  {
    
    vector<double> temp;
    ifstream fin(d.nomefile);
    if (!fin)
    {
      cout << "Errore di lettura di:" << d.nomefile << endl;
      return 1;
    }
    while (fin >> t)
    {
      temp.push_back(t * 10);
    }
    
    for (int i = 0; i <= temp.size() - 3; i = i + 3)
    {
      d.media_2_acq.push_back(media(temp, i, i + 3));
      d.dstd_2_acq.push_back(dstd_media(temp, i, i + 3));
    }
    
    for (int i = 0; i < d.dstd_2_acq.size(); i++)
    {
      if (d.dstd_2_acq[i] < sigma_dist_uni(10, 10))
      {
        d.dstd_2_acq[i] = sigma_dist_uni(10, 10);
      }
    }
    double delta_forza = kgpeso_to_newton((d.gp - 200.0) * 4 / 1000);
    d.media = media_ponderata(d.media_2_acq, d.dstd_2_acq);
    d.dstd_media = errore_media_ponderata(d.dstd_2_acq);
    d.kappa = d.media / delta_forza;
    d.sigma_kappa = sqrt(pow(1 / delta_forza, 2) * (pow(d.dstd_media, 2) + pow(sigma_dist_uni(10, 10), 2)) + 2 * pow(d.media * sigma_delta_forza / pow(delta_forza, 2), 2));
    if (d.nomefile.find("alto") != std::string::npos)
    {
      k_alto.push_back(d.kappa);
      sigma_k_alto.push_back(d.sigma_kappa);
      cout << "Alto"
           << "\t" << d.kappa << "\t" << d.sigma_kappa << "\t" << d.nomefile << endl;
    }
    else if (d.nomefile.find("basso") != std::string::npos)
    {
      k_basso.push_back(d.kappa);
      sigma_k_basso.push_back(d.sigma_kappa);
      cout << "Basso"
           << "\t" << d.kappa << "\t" << d.sigma_kappa << "\t" << d.nomefile << endl;
    }
  }

  double delta_1000 = abs(campioni[3].media - campioni[2].media);
  double delta_400 = abs(campioni[0].media - campioni[1].media);
  double delta_f = kgpeso_to_newton((1000.0 - 400.0) * 4 / 1000.0);
  double errore_desalva = (delta_1000 - delta_400) / delta_f;
  double num = campioni[3].media - campioni[1].media + campioni[2].media - campioni[0].media;
  double sigma_errore_desalva = sqrt(pow(1 / delta_f, 2) * (pow(campioni[0].dstd_media, 2) + pow(campioni[1].dstd_media, 2) + pow(campioni[2].dstd_media, 2) + pow(campioni[3].dstd_media, 2)) + 2 * pow(num * sigma_delta_forza / pow(delta_f, 2), 2));
  cout << "Errore DeSalvador: " << errore_desalva << "+/-" << sigma_errore_desalva << "micron/Newton" << endl;
  
  vector<double> k;
  vector<double> sigma_k;

  
  double media_ponderata_k_alto = media_ponderata(k_alto, sigma_k_alto); 
  k.push_back(media_ponderata_k_alto);
  double sigma_media_ponderata_k_alto = errore_media_ponderata(sigma_k_alto); 
  sigma_k.push_back(sigma_media_ponderata_k_alto);
  double compatib = abs(k_alto[0] - k_alto[1]) / sqrt(pow(sigma_k_alto[0], 2) + pow(sigma_k_alto[1], 2));
  cout << "K_alto: " << compatib << "\t" << media_ponderata_k_alto << "\t" << sigma_media_ponderata_k_alto << endl;
  
  
  double media_ponderata_k_basso = media_ponderata(k_basso, sigma_k_basso);
  k.push_back(media_ponderata_k_basso);
  double sigma_media_ponderata_k_basso = errore_media_ponderata(sigma_k_basso);
  sigma_k.push_back(sigma_media_ponderata_k_basso);
  double compatib2 = abs(k_basso[0] - k_basso[1]) / sqrt(pow(sigma_k_basso[0], 2) + pow(sigma_k_basso[1], 2));
  cout << "K_basso: " << compatib2 << "\t" << media_ponderata_k_basso << "\t" << sigma_media_ponderata_k_basso << endl;
  
  
  double valore_k = media_ponderata(k, sigma_k);
  double sigma_valore_k = errore_media_ponderata(sigma_k);

  cout << "K ponderato: " << valore_k << " +/- " << sigma_valore_k << " micron/Newton" << endl;
  cout << "Compatibilità K all e K acc tramite i singoli K: " << comp_3(media_ponderata_k_alto, media_ponderata_k_basso, sigma_media_ponderata_k_alto, sigma_media_ponderata_k_basso) << endl;

  
  cout << "SECONDO METODO CON CHI " << endl;
  vector<double> chi_basso; 
  vector<double> chi_alto;  
  vector<double> sigma_chi_basso;
  vector<double> sigma_chi_alto;
  vector<double> delta_forza_newton;

  for (int i = 0; i < campioni[0].media_2_acq.size(); i++)
  {
    delta_forza_newton.push_back(kgpeso_to_newton((200.0) * 4 / 1000));
  }
  for (int i = 0; i < campioni[2].media_2_acq.size(); i++) 
  {
    delta_forza_newton.push_back(kgpeso_to_newton((800.0) * 4 / 1000));
  }

  for (int i = 0; i < campioni[0].media_2_acq.size(); i++) 
  {
    chi_basso.push_back(campioni[0].media_2_acq[i]);
    chi_alto.push_back(campioni[1].media_2_acq[i]);
    sigma_chi_basso.push_back(campioni[0].dstd_2_acq[i]);
    sigma_chi_alto.push_back(campioni[1].dstd_2_acq[i]);
  }

  for (int i = 0; i < campioni[2].media_2_acq.size(); i++) 
  {
    chi_basso.push_back(campioni[2].media_2_acq[i]);
    chi_alto.push_back(campioni[3].media_2_acq[i]);
    sigma_chi_basso.push_back(campioni[2].dstd_2_acq[i]);
    sigma_chi_alto.push_back(campioni[3].dstd_2_acq[i]);
  }

  cout << "Angolare Basso:\t" << b_angolare(delta_forza_newton, chi_basso) << " +/- " << sigma_b(delta_forza_newton, chi_basso) << endl;
  cout << "Intercetta Basso:\t" << a_intercetta(delta_forza_newton, chi_basso) << " +/- " << sigma_a(delta_forza_newton, chi_basso) << endl;
  cout << "Angolare Alto:\t" << b_angolare(delta_forza_newton, chi_alto) << " +/- " << sigma_b(delta_forza_newton, chi_alto) << endl;
  cout << "Intercetta Alto:\t" << a_intercetta(delta_forza_newton, chi_alto) << " +/- " << sigma_a(delta_forza_newton, chi_alto) << endl;
  vector<double> chi;
  vector<double> intercetta;
  chi.push_back(b_angolare(delta_forza_newton, chi_basso));
  chi.push_back(b_angolare(delta_forza_newton, chi_alto));
  intercetta.push_back(a_intercetta(delta_forza_newton, chi_alto));
  intercetta.push_back(a_intercetta(delta_forza_newton, chi_basso));

  vector<double> sigma_chi;
  vector<double> sigma_intercetta;

  sigma_chi.push_back(sigma_b(delta_forza_newton, chi_basso));
  sigma_chi.push_back(sigma_b(delta_forza_newton, chi_alto));
  sigma_intercetta.push_back(sigma_b(delta_forza_newton, chi_alto));
  sigma_intercetta.push_back(sigma_b(delta_forza_newton, chi_basso));

  cout << "Compatibi K alto e k basso tramite chi quadro: " << comp_3(chi[0], chi[1], sigma_chi[0], sigma_chi[1]) << endl;
  cout << "Comp intercette: " << comp_3(a_intercetta(delta_forza_newton, chi_basso), a_intercetta(delta_forza_newton, chi_alto), sigma_a(delta_forza_newton, chi_basso), sigma_a(delta_forza_newton, chi_alto)) << endl;
  
  cout << "K con chi quadro tramite intepolazione 8 punti media pon k alto e K basso):\t" << media_ponderata(chi, sigma_chi) << " +/- " << errore_media_ponderata(sigma_chi) << endl;
  cout << "Intercetta pond: " << media_ponderata(intercetta, sigma_intercetta) << "+/-" << errore_media_ponderata(sigma_intercetta) << endl;

  return 0;
}

double kgpeso_to_newton(double kg_peso)
{
  return 9.806 * kg_peso;
}
\end{lstlisting}

\begin{lstlisting}[language=C++, label=lst:seconda_parte, caption=seconda\_parte]
#include <iostream>
#include <fstream>
#include <iomanip>
#include <cmath>
#include <string>
#include <vector>
#include "statistica2.h"
using namespace std;

struct estensimetri
{
	int n_est;
	double lunghezza;
	double diametro;
	double err_lunghezza = 2000.0; 
	double err_diametro_perce = 0.01;
	double err_diametro;
	double sezione;
	double intercetta_all;
	double coeff_ango_all;
	double intercetta_acc;
	double coeff_ango_acc;
	double err_intercetta_all;
	double err_coeff_ango_all;
	double err_intercetta_acc;
	double err_coeff_ango_acc;
	double k_medio;
	double err_k_medio;
	double int_medio;
	double err_int_medio;
	double err_intercetta_all_post;
	double err_intercetta_acc_post;
	double err_coeff_ango_all_post;
	double err_coeff_ango_acc_post;
	double compa_coeff_ango;
	double compa_intercetta;
	vector<double> allungamento;		   
	vector<double> accorciamento;		   
	vector<double> delta_x_all;			   
	vector<double> delta_x_acc;			   
	vector<double> err_delta_x_all;		   
	vector<double> err_delta_x_acc;		   
	vector<double> err_delta_x_all_priori; 
	vector<double> err_delta_x_acc_priori;
	vector<double> k_no_cons_all;
	vector<double> k_no_cons_acc;
	vector<double> err_k_no_cons_all;
	vector<double> err_k_no_cons_acc;
};

int main()
{
	int count = 0;
	double v, x0_all, x0_acc;
	vector<estensimetri> est(9); 
	vector<double> delta_f;
	vector<double> errori;
	vector<double> lunghezza_l;
	vector<double> k_medio_l;
	vector<double> err_k_medio_l;
	vector<double> reciproco_sezione;
	vector<double> k_medio_s;
	vector<double> err_k_medio_s;
	vector<double> rapporto;
	vector<double> prodotto;
	vector<double> err_rapporto;
	vector<double> err_prodotto;
	vector<double> diametro_quadrato;
	vector<double> young1;
	vector<double> err_young1;
	vector<double> young2;
	vector<double> err_young2;
	vector<double> young3_all_acc;
	vector<double> err_young3_all_acc;
	vector<double> e_1_metodo;
	vector<double> e_2_metodo;
	vector<double> e_3_metodo;
	vector<double> err_e_1_metodo;
	vector<double> err_e_2_metodo;
	vector<double> err_e_3_metodo;

	ifstream fin("set_dati.txt");
	if (!fin)
	{
		cout << "Errore lettura da file";
		return 1;
	}
	for (int i = 0; i < 9; i++)
	{
		est[i].n_est = i + 1;
		fin >> est[i].lunghezza;
		est[i].lunghezza = est[i].lunghezza * 1000.0; 
		fin >> est[i].diametro;
		est[i].diametro = est[i].diametro * 1000.0; 
	}
	while (fin >> v)
	{
		if (count % 2 == 0)
		{
			est[(count / 2) % 9].allungamento.push_back(v * 1e+6); 
		}
		else
		{
			est[(count / 2) % 9].accorciamento.push_back(v * 1e+6);
		}
		count++;
	}

	for (int i = 0; i < est.size(); i++) 
	{
		for (int j = 0; j < est[i].allungamento.size(); j++)
		{
			est[i].err_delta_x_all_priori.push_back(sqrt(2) * sigma_dist_uni(5.0, 1));
			est[i].err_delta_x_acc_priori.push_back(sqrt(2) * sigma_dist_uni(5.0, 1));
		}
	} 

	for (int j = 0; j < est[0].allungamento.size(); j++)
	{
		delta_f.push_back(0.1 * 4 * 9.806 * j);
	}

	for (int i = 0; i < est.size(); i++) 
	{
		x0_all = est[i].allungamento[0];
		x0_acc = est[i].accorciamento[0];

		for (int j = 0; j < 10; j++) 
		{
			est[i].delta_x_all.push_back(est[i].allungamento[j] - x0_all);
			est[i].delta_x_acc.push_back(est[i].accorciamento[j] - x0_acc);
		}
		double sigma_post_all = sigma_y_posteriori(delta_f, est[i].delta_x_all);
		double sigma_post_acc = sigma_y_posteriori(delta_f, est[i].delta_x_acc);

		cout<<est[i].n_est<<"\t"<<sigma_post_all<<"\t"<<sigma_post_acc<<"\t"<<sqrt(2) * sigma_dist_uni(5.0, 1)<<"\t"<<sqrt(2) * sigma_dist_uni(10.0, 1)<<endl;

		for (int j = 0; j < est[i].allungamento.size(); j++)
		{
			est[i].err_delta_x_all.push_back(sqrt(2) * sigma_post_all); 
			est[i].err_delta_x_acc.push_back(sqrt(2) * sigma_post_acc);
		}

		est[i].intercetta_all = a_intercetta(delta_f, est[i].delta_x_all);				  
		est[i].coeff_ango_all = b_angolare(delta_f, est[i].delta_x_all);				  
		est[i].intercetta_acc = a_intercetta(delta_f, est[i].delta_x_acc);				  
		est[i].coeff_ango_acc = b_angolare(delta_f, est[i].delta_x_acc);				  
		est[i].err_intercetta_all = sigma_a_y_uguali(delta_f, est[i].err_delta_x_all[0]); 
		est[i].err_coeff_ango_all = sigma_b_y_uguali(delta_f, est[i].err_delta_x_all[0]); 
		est[i].err_intercetta_acc = sigma_a_y_uguali(delta_f, est[i].err_delta_x_acc[0]); 
		est[i].err_coeff_ango_acc = sigma_b_y_uguali(delta_f, est[i].err_delta_x_acc[0]); 
		est[i].err_intercetta_all_post = sigma_a_y_uguali(delta_f, sigma_post_all);		  
		est[i].err_intercetta_acc_post = sigma_a_y_uguali(delta_f, sigma_post_acc);
		est[i].err_coeff_ango_all_post = sigma_b_y_uguali(delta_f, sigma_post_all);
		est[i].err_coeff_ango_acc_post = sigma_b_y_uguali(delta_f, sigma_post_acc);
		est[i].err_diametro = est[i].err_diametro_perce * est[i].diametro;
		est[i].compa_coeff_ango = comp_3(est[i].coeff_ango_acc, est[i].coeff_ango_all, est[i].err_coeff_ango_acc_post, est[i].err_coeff_ango_all_post); 
		est[i].compa_intercetta = comp_3(est[i].intercetta_acc, est[i].intercetta_all, est[i].err_intercetta_acc_post, est[i].err_intercetta_all_post);

		cout<<est[i].delta_x_acc.size()<<" "<<est[i].err_delta_x_all_priori.size()<<endl;
	}
	cout << "POST ACC:" << sigma_y_posteriori(delta_f, est[1].delta_x_acc) << endl;
	cout << "POST ALL:" << sigma_y_posteriori(delta_f, est[1].delta_x_all) << endl;

	for (int i = 0; i < est.size(); i++) 
	{
		vector<double> k_alto_basso;
		vector<double> err_k_alto_basso;
		vector<double> int_alto_basso;
		vector<double> err_int_alto_basso;

		k_alto_basso.push_back(est[i].coeff_ango_acc);
		k_alto_basso.push_back(est[i].coeff_ango_all);
		int_alto_basso.push_back(est[i].intercetta_acc);
		int_alto_basso.push_back(est[i].intercetta_all);
		err_k_alto_basso.push_back(est[i].err_coeff_ango_acc_post); 
		err_k_alto_basso.push_back(est[i].err_coeff_ango_all_post); 
		err_int_alto_basso.push_back(est[i].err_intercetta_acc_post);
		err_int_alto_basso.push_back(est[i].err_intercetta_all_post);

		est[i].k_medio = media_ponderata(k_alto_basso, err_k_alto_basso);
		est[i].err_k_medio = errore_media_ponderata(err_k_alto_basso);
		est[i].int_medio = media_ponderata(int_alto_basso, err_int_alto_basso);
		est[i].err_int_medio = errore_media_ponderata(err_int_alto_basso);
		est[i].sezione = M_PI * pow(((est[i].diametro) / 2.0), 2.0);

		ofstream gout("../Grafici_2/est_" + to_string(est[i].n_est) + ".txt");	 
		ofstream hout("../Latex/riepilogo_" + to_string(est[i].n_est) + ".txt"); 
		ofstream excelout("../estensimetri_per_excel.txt", ios::app);
		hout << "Numero:\t" << est[i].n_est << endl;
		hout << "Diametro:\t" << est[i].diametro << endl;
		hout << "Lunghezza\t" << est[i].lunghezza << endl;
		hout << "Ang. acc.\t" << est[i].coeff_ango_acc << " +/- " << est[i].err_coeff_ango_acc_post << endl;	
		hout << "Interc. acc.\t" << est[i].intercetta_acc << " +/- " << est[i].err_intercetta_acc_post << endl; 
		hout << "Ang. all.\t" << est[i].coeff_ango_all << "+/- " << est[i].err_coeff_ango_all_post << endl;		
		hout << "Interc. all.\t" << est[i].intercetta_all << " +/- " << est[i].err_intercetta_all_post << endl; 
		hout << "Media Pond \t" << est[i].k_medio << "+/-" << est[i].err_k_medio << endl;
		
		excelout << est[i].coeff_ango_acc << " +/- " << est[i].err_coeff_ango_acc_post << "\t" << est[i].coeff_ango_all << "+/- " << est[i].err_coeff_ango_all_post << "\t" << est[i].k_medio << "+/-" << est[i].err_k_medio << "\t" << comp_3(est[i].coeff_ango_acc, est[i].coeff_ango_all, est[i].err_coeff_ango_acc_post, est[i].err_coeff_ango_all_post) << "\t" << est[i].intercetta_acc << " +/- " << est[i].err_intercetta_acc_post << "\t" << est[i].intercetta_all << " +/- " << est[i].err_intercetta_all_post << "\t" << est[i].int_medio << " +/- " << est[i].err_int_medio << "\t" << comp_3(est[i].intercetta_acc, est[i].intercetta_all, est[i].err_intercetta_acc_post, est[i].err_intercetta_all_post) << endl;

		gout << "#DeltaF[Newton]\t#All[micron]\t#ErrAll[micron]\t#Acc[micron]\t#ErrAcc[micron]\tsigma_post_ALL\tsigma_post_ACC" << endl;

		for (int k = 0; k < est[i].accorciamento.size(); k++) 
		{													  

			gout << delta_f[k] << "\t" << est[i].delta_x_all[k] << "\t" << est[i].err_delta_x_all[k] << "\t" << est[i].delta_x_acc[k] << "\t" << est[i].err_delta_x_acc[k] << "\t" << sigma_y_posteriori(delta_f, est[i].delta_x_all) << "\t" << sigma_y_posteriori(delta_f, est[i].delta_x_acc) << endl;
		}
	}
	
	for (int i; i < est.size(); i++)
	{
		double sum_chi=0;
		for(int j=0; j<est[i].delta_x_acc.size(); j++){
			sum_chi=sum_chi+pow(((est[i].delta_x_acc[j]-a_intercetta(delta_f, est[i].delta_x_acc)-(b_angolare(delta_f, est[i].delta_x_acc)*delta_f[j]))/est[i].err_delta_x_acc_priori[j]),2);
			cout<<est[i].err_delta_x_acc_priori[j]<<endl;
		}
		cout<<i<<":"<<sum_chi<<endl;
	}

	estensimetri nos_est; 
	nos_est.n_est = 10;
	nos_est.lunghezza = 700000.0;
	nos_est.diametro = 279;
	nos_est.allungamento = {-2.33333, 223.333, 430.667, 660, 877, 1092, 1310.33, 1512.33, 1748.67, 1948.33, 2170};
	nos_est.accorciamento = {1.66667, 219.667, 430.333, 655, 869, 1081.67, 1303.67, 1513.67, 1727, 1949.67, 2170};
	nos_est.err_diametro = nos_est.err_diametro_perce * nos_est.diametro;
	nos_est.sezione = M_PI * pow((nos_est.diametro / 2.0), 2);
	vector<double> delta_f_nos;
	for (int i = 0; i < nos_est.allungamento.size(); i++)
	{
		delta_f_nos.push_back(0.1 * 4 * 9.806 * i);
		nos_est.delta_x_all.push_back(nos_est.allungamento[i] - nos_est.allungamento[0]);
		nos_est.delta_x_acc.push_back(nos_est.accorciamento[i] - nos_est.accorciamento[0]);
	}
	for (int h = 0; h < nos_est.allungamento.size(); h++)
	{
		nos_est.err_delta_x_all.push_back(sqrt(2) * sigma_y_posteriori(delta_f_nos, nos_est.delta_x_all)); 
		nos_est.err_delta_x_acc.push_back(sqrt(2) * sigma_y_posteriori(delta_f_nos, nos_est.delta_x_all));
	}
	double sigma_post_all = sigma_y_posteriori(delta_f_nos, nos_est.delta_x_all);
	double sigma_post_acc = sigma_y_posteriori(delta_f_nos, nos_est.delta_x_acc);

	nos_est.intercetta_all = a_intercetta(delta_f_nos, nos_est.delta_x_all);				
	nos_est.coeff_ango_all = b_angolare(delta_f_nos, nos_est.delta_x_all);					
	nos_est.intercetta_acc = a_intercetta(delta_f_nos, nos_est.delta_x_acc);				
	nos_est.coeff_ango_acc = b_angolare(delta_f_nos, nos_est.delta_x_acc);					
	nos_est.err_intercetta_all = sigma_a_y_uguali(delta_f_nos, nos_est.err_delta_x_all[0]); 
	nos_est.err_coeff_ango_all = sigma_b_y_uguali(delta_f_nos, nos_est.err_delta_x_all[0]); 
	nos_est.err_intercetta_acc = sigma_a_y_uguali(delta_f_nos, nos_est.err_delta_x_acc[0]); 
	nos_est.err_coeff_ango_acc = sigma_b_y_uguali(delta_f_nos, nos_est.err_delta_x_acc[0]); 
	nos_est.err_intercetta_all_post = sigma_a_y_uguali(delta_f_nos, sigma_post_all);		
	nos_est.err_intercetta_acc_post = sigma_a_y_uguali(delta_f_nos, sigma_post_acc);
	nos_est.err_coeff_ango_all_post = sigma_b_y_uguali(delta_f_nos, sigma_post_all);
	nos_est.err_coeff_ango_acc_post = sigma_b_y_uguali(delta_f_nos, sigma_post_acc);
	nos_est.err_diametro = nos_est.err_diametro_perce * nos_est.diametro;
	nos_est.compa_coeff_ango = comp_3(nos_est.coeff_ango_acc, nos_est.coeff_ango_all, nos_est.err_coeff_ango_acc_post, nos_est.err_coeff_ango_all_post); 
	nos_est.compa_intercetta = comp_3(nos_est.intercetta_acc, nos_est.intercetta_all, nos_est.err_intercetta_acc_post, nos_est.err_intercetta_all_post);

	vector<double> k_alto_basso_nos;
	vector<double> err_k_alto_basso_nos;

	k_alto_basso_nos.push_back(nos_est.coeff_ango_acc);
	k_alto_basso_nos.push_back(nos_est.coeff_ango_all);
	err_k_alto_basso_nos.push_back(nos_est.err_coeff_ango_acc_post); 
	err_k_alto_basso_nos.push_back(nos_est.err_coeff_ango_all_post); 

	nos_est.k_medio = media_ponderata(k_alto_basso_nos, err_k_alto_basso_nos);
	nos_est.err_k_medio = errore_media_ponderata(err_k_alto_basso_nos);
	cout << nos_est.n_est << "\t" << nos_est.err_delta_x_all[0] << "\t" << nos_est.err_delta_x_acc[0] << endl;

	est.push_back(nos_est); 

	for (int i = 0; i < est.size(); i++)
	{
		
		
		if (est[i].diametro == 279) 
		{
			lunghezza_l.push_back(est[i].lunghezza);
			k_medio_l.push_back(est[i].k_medio);
			err_k_medio_l.push_back(est[i].err_k_medio);
			rapporto.push_back(est[i].k_medio / est[i].lunghezza);
			err_rapporto.push_back(sqrt(pow((est[i].err_k_medio / est[i].lunghezza), 2) + pow(((est[i].k_medio * est[i].err_lunghezza) / pow(est[i].lunghezza, 2)), 2)));
			
		}

		if (est[i].lunghezza == 950000) 
		{
			reciproco_sezione.push_back(1 / est[i].sezione);
			diametro_quadrato.push_back(est[i].diametro * est[i].diametro); 
			k_medio_s.push_back(est[i].k_medio);
			err_k_medio_s.push_back(est[i].err_k_medio);
			prodotto.push_back(est[i].k_medio * pow(est[i].diametro, 2));
			err_prodotto.push_back(sqrt(pow(pow(est[i].diametro, 2) * est[i].err_k_medio, 2) + (pow((2 * est[i].k_medio * est[i].diametro), 2) * pow(est[i].err_diametro, 2))));
		}
	}

	
	ofstream scost("../Grafici_2/a_sez_cost.txt");
	ofstream lcost("../Grafici_2/a_lung_cost.txt");
	ofstream rcost("../Grafici_2/rapporto.txt");
	ofstream pcost("../Grafici_2/prodotto.txt");
	
	
	cout << "A SEZ COST (lunghezza in asse x)" << endl;
	cout<<"ERR SEZIONE:"<<(8/(M_PI*))
	cout << "Coeff. ango.\t" << b_angolare(lunghezza_l, k_medio_l) << " +/- " << sigma_b_y_uguali(lunghezza_l, err_k_medio_l[0]) << endl;
	cout << "Intercetta\t" << a_intercetta(lunghezza_l, k_medio_l) << " +/- " << sigma_a_y_uguali(lunghezza_l, err_k_medio_l[0]) << endl;
	scost << "#LunghezzaEst[mm]\t#K[micron/N]\t#Err[micron/N]" << endl;
	rcost << "#Lunghezza[mm]\t#Rapporto[1/N]\t#Err[1/N]" << endl;

	for (int o = 0; o < lunghezza_l.size(); o++)
	{
		scost << lunghezza_l[o] << "\t" << k_medio_l[o] << "\t" << err_k_medio_l[o] << endl;
		rcost << lunghezza_l[o] << "\t" << rapporto[o] << "\t" << err_rapporto[o] << endl;
	}
	
	cout << "A LUNGHEZZA COST (1/sezione su asse x)" << endl;
	cout << "Coeff. ango.\t" << b_angolare(reciproco_sezione, k_medio_s) << " +/- " << sigma_b_y_uguali(reciproco_sezione, err_k_medio_l[0]) << endl;
	cout << "Intercetta\t" << a_intercetta(reciproco_sezione, k_medio_s) << " +/- " << sigma_a_y_uguali(reciproco_sezione, err_k_medio_l[0]) << endl;
	lcost << "#ReciprocoSez[1/micron]\t#K[micron/N]\t#Err[micron/N]" << endl;
	pcost << "#DiametroQuadrato[micron^2]\t#Prodotto[micron^3/N]\t#Err[micron^3/N]" << endl;
	for (int o = 0; o < reciproco_sezione.size(); o++)
	{
		lcost << reciproco_sezione[o] << "\t" << k_medio_s[o] << "\t" << err_k_medio_s[o] << "\t" << endl;
		pcost << diametro_quadrato[o] << "\t" << prodotto[o] << "\t" << err_prodotto[o] << endl;
	}
	

	
	ofstream eout1("../CalcoloE/primo_metodo.txt");
	ofstream eout2("../CalcoloE/secondo_metodo.txt");
	ofstream eout3("../CalcoloE/terzo_metodo.txt");

	
	for (int i = 0; i < est.size(); i++)
	{
		vector<double> young_i;
		vector<double> err_young_i;
		young_i.push_back(4 * est[i].lunghezza / (M_PI * pow(est[i].diametro, 2) * est[i].coeff_ango_all));
		young_i.push_back(4 * est[i].lunghezza / (M_PI * pow(est[i].diametro, 2) * est[i].coeff_ango_acc));
		err_young_i.push_back(young_i[0] * sqrt(pow((est[i].err_lunghezza / est[i].lunghezza), 2) + pow((est[i].err_coeff_ango_all / est[i].coeff_ango_all), 2) + 4 * pow((est[i].err_diametro / est[i].diametro), 2))); 
		err_young_i.push_back(young_i[1] * sqrt(pow((est[i].err_lunghezza / est[i].lunghezza), 2) + pow((est[i].err_coeff_ango_acc / est[i].coeff_ango_acc), 2) + 4 * pow((est[i].err_diametro / est[i].diametro), 2)));
		young1.push_back(media_ponderata(young_i, err_young_i));
		err_young1.push_back(errore_media_ponderata(err_young_i));
	}
	
	for (int i = 0; i < est.size(); i++)
	{
		young2.push_back(4 * est[i].lunghezza / (M_PI * pow(est[i].diametro, 2) * est[i].k_medio));
		err_young2.push_back(young2[i] * sqrt(pow((est[i].err_lunghezza / est[i].lunghezza), 2) + pow((est[i].err_k_medio / est[i].k_medio), 2) + 4 * pow((est[i].err_diametro / est[i].diametro), 2)));
	}
	eout1 << "#Nest\t#E[N/micron^2]\t#ErrE[N/micron^2]" << endl;
	eout2 << "#Nest\t#E[N/micron^2]\t#ErrE[N/micron^2]" << endl;
	eout3 << "#Nest\t#E[N/micron^2]\t#ErrE[N/micron^2]" << endl;
	for (int i = 0; i < est.size(); i++)
	{
		eout1 << est[i].n_est << "\t" << young1[i] << "\t" << err_young1[i] << endl;
		eout2 << est[i].n_est << "\t" << young2[i] << "\t" << err_young2[i] << endl;
		if (young1[i] < 0.3) 
		{
			e_1_metodo.push_back(young1[i]);
			err_e_1_metodo.push_back(err_young1[i]);
			e_2_metodo.push_back(young2[i]);
			err_e_2_metodo.push_back(err_young2[i]);
		}
	}
	
	for (int i = 0; i < est.size(); i++)
	{
		double upper_bound;
		double sigma_delta_f = sigma_dist_uni((((300.0 * 100.0) / (11.0 * 1000.0)) * (4.0 * 9.806 / 1000.0)), 1.0);
		if (i == est.size() - 1) 
		{
			upper_bound = est[i].allungamento.size() - 1;
		}
		else
		{
			upper_bound = est[i].allungamento.size();
		}
		for (int j = 0; j < upper_bound; j++)
		{
			if (j % 2 == 0)
			{
				est[i].k_no_cons_all.push_back(abs((est[i].allungamento[j + 1] - est[i].allungamento[j])) / (0.4 * 9.806));
				est[i].k_no_cons_acc.push_back(abs((est[i].accorciamento[j + 1] - est[i].accorciamento[j])) / (0.4 * 9.806));
				est[i].err_k_no_cons_all.push_back(sqrt(pow(1 / (0.4 * 9.806), 2) * pow(est[i].err_delta_x_all[j], 2) + pow((-(est[i].allungamento[j + 1] - est[i].allungamento[j]) / pow((0.4 * 9.806), 2)), 2) * pow(sigma_delta_f, 2)));
				est[i].err_k_no_cons_acc.push_back(sqrt(pow(1 / (0.4 * 9.806), 2) * pow(est[i].err_delta_x_acc[j], 2) + pow((-(est[i].accorciamento[j + 1] - est[i].accorciamento[j]) / pow((0.4 * 9.806), 2)), 2) * pow(sigma_delta_f, 2))); 
			}
		}
		double err_1 = est[i].err_lunghezza / est[i].lunghezza;
		double err_2_acc = dstd_media(est[i].k_no_cons_acc) / media(est[i].k_no_cons_acc);
		double err_2_all = dstd_media(est[i].k_no_cons_all) / media(est[i].k_no_cons_all);
		double err_3 = 2 * est[i].err_diametro / est[i].diametro;
		vector<double> young3;
		vector<double> err_young3;

		double young3_all = est[i].lunghezza / (est[i].sezione * media(est[i].k_no_cons_all));
		double young3_acc = est[i].lunghezza / (est[i].sezione * media(est[i].k_no_cons_acc));
		double sigma_young_3_all = young3_all * sqrt(pow(err_1, 2) + pow(err_2_all, 2) + pow(err_3, 2));
		double sigma_young_3_acc = young3_acc * sqrt(pow(err_1, 2) + pow(err_2_acc, 2) + pow(err_3, 2));
		young3.push_back(young3_all);
		young3.push_back(young3_acc);
		err_young3.push_back(sigma_young_3_acc);
		err_young3.push_back(sigma_young_3_all);

		//cout << young3_all << "\t" << sigma_young_3_all << "\t" << young3_acc << "\t" << sigma_young_3_acc << endl; //Stampa modulo young in all e acc
		young3_all_acc.push_back(media_ponderata(young3, err_young3));
		err_young3_all_acc.push_back(errore_media_ponderata(err_young3));
	}
	for (int k = 0; k < young3_all_acc.size(); k++)
	{
		//cout << young3_all_acc[i] << "\t" << err_young3_all_acc[i] << endl;
		eout3 << est[k].n_est << "\t" << young3_all_acc[k] << "\t" << err_young3_all_acc[k] << endl;
		if (young3_all_acc[k] < 0.3) 
		{
			e_3_metodo.push_back(young3_all_acc[k]);
			err_e_3_metodo.push_back(err_young3_all_acc[k]);
		}
	}

	for (int i = 0; i < est.size() - 1; i++)
	{
		cout << est[i].n_est << "\t" << sigma_y_posteriori(delta_f, est[i].delta_x_all) << "\t" << sigma_y_posteriori(delta_f, est[i].delta_x_acc) << endl;
	}
	cout << est[9].n_est << "\t" << sigma_y_posteriori(delta_f_nos, est[9].delta_x_all) << "\t" << sigma_y_posteriori(delta_f_nos, est[9].delta_x_acc) << endl;
	double m_1_met = media(e_1_metodo);
	double m_2_met = media(e_2_metodo);
	double m_3_met = media(e_3_metodo);
	double err_m_1_met = dstd_media(e_1_metodo);
	double err_m_2_met = dstd_media(e_2_metodo);
	double err_m_3_met = dstd_media(e_3_metodo);
	cout << "Media pond 1 metodo reiettati: " << media_ponderata(e_1_metodo, err_e_1_metodo) << "+/-" << errore_media_ponderata(err_e_1_metodo) << endl;
	cout << "Media pond 2 metodo reiettati: " << media_ponderata(e_2_metodo, err_e_2_metodo) << "+/-" << errore_media_ponderata(err_e_2_metodo) << endl;
	cout << "Media pond 3 metodo reiettati: " << media_ponderata(e_3_metodo, err_e_3_metodo) << "+/-" << errore_media_ponderata(err_e_3_metodo) << endl;
	cout << "Media 1 metodo reiettati: " << media(e_1_metodo) << "+/-" << dstd_media(e_1_metodo) << endl;
	cout << "Media 2 metodo reiettati: " << media(e_2_metodo) << "+/-" << dstd_media(e_2_metodo) << endl;
	cout << "Media 3 metodo reiettati: " << media(e_3_metodo) << "+/-" << dstd_media(e_3_metodo) << endl;
	cout << "Comp 1 2: " << comp_3(m_1_met, m_2_met, err_m_1_met, err_m_2_met) << endl;
	cout << "Comp 1 3: " << comp_3(m_1_met, m_3_met, err_m_1_met, err_m_3_met) << endl;
	cout << "Comp 2 3: " << comp_3(m_3_met, m_2_met, err_m_3_met, err_m_2_met) << endl;
	
	return 0;
}
\end{lstlisting}

\end{document}*
